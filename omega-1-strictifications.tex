\documentclass[12pt]{article}
\usepackage[hyphens]{url} % 'hyphens' option allows line breaks after "-" characters
\usepackage[colorlinks,allcolors=blue]{hyperref}
\usepackage{cite}
\usepackage{dutchcal}
\usepackage{amssymb}
\usepackage{amsthm}
\usepackage{amsfonts}
\usepackage{ dsfont }
\usepackage{mathtools}
\usepackage{amsmath}
\usepackage[all,cmtip]{xy}
\usepackage{fullpage}
\usepackage{mathrsfs}
\usepackage{graphicx}
\usepackage{tensor}
\usepackage{tikz}
\usepackage{tikz-cd}
\usetikzlibrary{matrix}
\newtheorem{theorem}{Theorem}[section]
\newtheorem{lemma}[theorem]{Lemma}
\newtheorem{exercise}[theorem]{Exercise}
\newtheorem{corollary}[theorem]{Corollary}
\newtheorem{claim}[theorem]{Claim}
\newtheorem{fact}[theorem]{Fact}
\newtheorem{proposition}[theorem]{Proposition}

\newtheorem{conjecture}[theorem]{Conjecture}

\theoremstyle{definition}
\newtheorem{definition}[theorem]{Definition}


\newenvironment{example}[1][Example]{\begin{trivlist}
\item[\hskip \labelsep {\bfseries #1}]}{\end{trivlist}}
\newenvironment{remark}[1][Remark]{\begin{trivlist}
\item[\hskip \labelsep {\bfseries #1}]}{\end{trivlist}}
\newenvironment{moral}[1][Moral]{\begin{trivlist}
\item[\hskip \labelsep {\bfseries #1}]}{\end{trivlist}}
\newcommand{\Z}{\mathbb{Z}}
\newcommand{\Q}{\mathbb{Q}}
\newcommand{\R}{\mathbb{R}}
\newcommand{\N}{\mathbb{N}}

\newcommand{\TODO}[1]{\textcolor{red}{TODO: {#1}}}
\newcommand{\feedback}[1]{\textcolor{blue}{FEEDBACK REQUEST: {#1}}}

\newcommand{\T}{\mathcal{T}}
\renewcommand{\S}{\mathcal{S}}

\newcommand{\eps}{\varepsilon}
\renewcommand{\phi}{\varphi}


% names of categories
\newcommand{\C}{\mathcal{C}}
\newcommand{\D}{\mathcal{D}}
\newcommand{\V}{\mathcal{V}}
\newcommand{\W}{\mathcal{W}}

\newcommand{\sset}{\text{sSet}}
\newcommand{\stinfty}{\omega\text{Gpd}}
\newcommand{\algkan}{\text{AlgKan}}
\newcommand{\stcom}{\text{sTCom}}
\newcommand{\crcom}{\text{CrCom}}
\newcommand{\ch}{\text{Ch}_\mathbb{Z}^+}
\newcommand{\sabgroup}{\text{sAbGrp}}
\newcommand{\qsset}{\mathcal{sSet}}
\newcommand{\qstcom}{\mathcal{sTCom}}
\newcommand{\grpd}{\text{Grpd}}
%newer categories (this paper)
\newcommand{\omegacat}{\omega\mathcal{Cat}}
\newcommand{\omegancat}[1]{(\omega,#1)\mathcal{Cat}}
\newcommand{\graycat}[1]{((\omega, #1)\mathcal{Cat})^\otimes{-}\mathcal{Cat}}
\newcommand{\cartesiancat}[1]{((\omega, #1)\mathcal{Cat})^\times{-}\mathcal{Cat}}
\newcommand{\graycatzero}{(\stinfty)^\otimes\mathcal{Cat}}
\newcommand{\cartesiancatzero}{(\stinfty)^\times\mathcal{Cat}}

\newcommand{\cartcrossedcat}{\crcom^\times\text{-}\mathcal{Cat}}
\newcommand{\tensorcrossedcat}{\crcom^\otimes\text{-}\mathcal{Cat}}
\newcommand{\tencart}[1]{\underline{#1}}
\newcommand{\grpdcat}{\text{Grpd}\text{-}\mathcal{Cat}}

\newcommand{\joyal}{\text{sSet}_{\mathcal{J}}}
\newcommand{\bergner}{(\text{sSet})-\text{Cat}_{\mathcal{B}}}
\newcommand{\ssetcat}{\text{sSet}\text{-}\mathcal{Cat}}




%misc
\newcommand{\id}{\text{id}}
\newcommand{\Id}{\text{Id}}
\newcommand{\colim}{\emph{colim}}
\newcommand{\holim}{\text{holim}}
\newcommand{\dec}{\text{dec}}
\newcommand{\ob}{\text{ob}}
\newcommand{\del}{\partial}
\newcommand{\Ho}{\text{Ho}}
\newcommand{\ho}{\text{ho}}
\newcommand{\sus}{\Sigma}

% names of functors. 
\newcommand{\stalg}{\text{St}_{\text{Alg}}}
\newcommand{\ualg}{\text{U}_{\text{Alg}}}
\newcommand{\st}{\text{St}}
\newcommand{\ust}{\text{U}_{\st}}
\newcommand{\Tot}{\text{Tot}}
\newcommand{\abtcom}{\text{Ab}_{\text{sT}}}
\newcommand{\utcom}{\text{U}_{\text{sT}}}
\newcommand{\abcomplex}{\text{Ab}_{\text{Cr}}}
\newcommand{\ucomplex}{\text{U}_{\text{Cr}}}
\newcommand{\doldnormalizer}{\text{N}_\mathbb{Z}}
\newcommand{\doldsset}{\Gamma_\mathbb{Z}}
\newcommand{\nonabdoldnormalizer}{\text{N}_{\crcom}}
\newcommand{\nonabdoldsset}{\Gamma_{\crcom}}
\newcommand{\lf}{\mathcal{L}}
\newcommand{\rf}{\mathcal{R}}
\newcommand{\tz}{\tilde{\mathbb{Z}}}
\newcommand{\tabcomplex}{\tilde{\text{Ab}}_{\text{Cr}}}
%new this paper
\newcommand{\strictification}[1]{\text{St}_{#1}}
\newcommand{\ninclusion}[1]{U_{#1}}
\newcommand{\omegannerve}[1]{N_{(\omega,#1)}}
\newcommand{\omeganrigidification}[1]{C_{(\omega,#1)}}
\newcommand{\forgetcartesian}{R}
\newcommand{\forcecartesian}{L}
\newcommand{\leftone}{\st^{\text{loc}}}
\newcommand{\rightone}{U^{\text{loc}}}
\newcommand{\lefttwo}{\st^{\text{glo}}}
\newcommand{\righttwo}{U^{\text{glo}}}
\newcommand{\freeatom}[1]{\mathcal{F}\langle \text{at}{#1}\rangle}
\newcommand{\takeapart}{\lambda}
\newcommand{\puttogether}{\gamma}
\newcommand{\hotwo}{\ho_{(2,1)}}


\newcommand{\freecoalg}{\mathcal{F}_{\st}}
\renewcommand{\dim}[1]{\operatorname{dim}\mleft({#1}\mright)}
\renewcommand{\det}[1]{\operatorname{det}\mleft({#1}\mright)}
\renewcommand{\gcd}[1]{\operatorname{gcd}\mleft\{{#1}\mright\}}
\renewcommand{\min}[1]{\operatorname{min}\mleft\{{#1}\mright\}}
\renewcommand{\max}[1]{\operatorname{max}\mleft\{{#1}\mright\}}
\DeclareMathOperator{\Hom}{Hom}
\DeclareMathOperator{\Aut}{Aut}
\newcommand{\on}{\operatorname}

\setcounter{tocdepth}{1}
%-------------------------------------------------------------------------------
\graphicspath{{images/}}
%-------------------------------------------------------------------------------
\begin{document}
%-------------------------------------------------------------------------------
\author{Kimball Strong}
\date{}

%-------------------------------------------------------------------------------
\title{\vspace{-1in} On the Strictification of $(\infty,1)$-categories\\ \vspace{.1in}}

\maketitle
%-------------------------------------------------------------------------------
\section{Introduction}	
	\TODO{ask tim porter if I'm neglecting any references. Good excuse to get some eyes on it.}
	The singular homology of a space is one of the largest successes of algebraic topology: it is strong enough to distinguish many spaces from each other, yet frequently very computable. 
	One of the most useful theorems in the day-to-day life of an algebraic topologist is the \textit{Homological Whitehead Theorem}, which asserts that homology is sufficient to detect whether or not a given map between simply connected spaces is a weak equivalence:
	\begin{theorem}
		Let $f:X \to Y$ be a map of simply connected CW complexes. Then if $f$ induces isomorphisms $H_n(X) \to H_n(Y)$ for all $n$, $f$ is a homotopy equivalence. 
	\end{theorem}
	Thus the study of whether two spaces can be continuously deformed into each other is partially reduced to calculating certain maps between abelian groups. This theorem really comes about as the combination of two theorems: firstly, a homotopical piece:
	\begin{theorem}[Whitehead Theorem]
		Let $f: X \to Y$ be a map of CW-complexes. Then if $f$ induces isomorphisms $\pi_n(X) \to \pi_n(Y)$ for all $n$ (and, for $n > 0$, all choices of basepoints) then $f$ is a homotopy equivalence.
	\end{theorem}
	Along with the homological content:
	\begin{theorem}[Homological Whitehead Theorem]
		Let $f: X \to Y$ be a map of simply connected CW complexes. Then if $f$ induces isomorphisms $H_n(X) \to H_n(Y)$ for all $n$, $f$ also induces isomorphisms $\pi_n(X) \to \pi_n(Y)$ (for any choice of basepoint).
	\end{theorem}
	Zooming out to the categorical level, we can rephrase this as saying the singular chains functor reflects weak homotopy equivalences between simply-connected $CW$-complexes. 
	\par 
	A primary defect of this theorem is that it requires simple connectedness, and this requirement cannot be weakened; there are so-called ``acyclic'' CW complexes $X$ for which the unique map $X \to \bullet$ is a homology equivalence, but $X$ is not homotopy equivalent to a point. 
	Fortunately, it turns out that by appropriately generalizing simply-connected chain complexes, we can find a stronger statement:
	\begin{theorem}
		Let $f: X \to Y$ be a map of CW complexes. If
		\begin{itemize}
			\item $f$ induces an isomorphism on $\pi_0$ , and
			\item for any choice of basepoint, $f$ induces an isomorphism on $\pi_1$, and
			\item for any choice of basepoint, the induced map of universal covers $\widehat{f}: \widehat{X} \to \widehat{Y}$ induces an isomorphism on homology.
		\end{itemize}
		Then $f$ is a weak equivalence.
	\end{theorem}
	The conditions in this theorem can be reinterpreted in a surprising way, in terms of a higher-dimensional generalization of a groupoid called an \textit{$\omega$-groupoid:}
	\begin{definition}
		An $\omega$-groupoid $G$ is a sequence of sets
		\begin{center}
		\begin{tikzcd}[sep = huge]
		G_0 & G_1 \ar[l, "s", shift left = 2] \ar[l,"t" swap, shift right = 2] & G_2 \ar[l, "s", shift left = 2] \ar[l,"t" swap, shift right = 2]   & \cdots \ar[l, "s", shift left = 2] \ar[l,"t" swap, shift right = 2] 
		\end{tikzcd}
		\end{center}
		Where each subdiagram
		\begin{center}
		\begin{tikzcd}[sep = huge]
		G_i & G_{k} \ar[l, "s^{k-i}", shift left = 2] \ar[l,"t^{k-i}" swap, shift right = 2]
		\end{tikzcd}
		\end{center}
		(with $i < k$) is equipped with a composition operation $\circ_i^k$ making it into a groupoid. Furthermore, these operations are compatible in the sense that if a diagram in $G$ can be reduced to a single $n$-cell via a sequence of compositions, then there is a unique way to do so. 
		For an omega groupoid $G$ and $g \in G_0$, we denote by $\pi_n(G,g)$ the automorphism group of $g$ in the groupoid
		\begin{center}
		\begin{tikzcd}[sep = huge]
		G_0 & G_{n} \ar[l, "s^n", shift left = 2] \ar[l,"t^n" swap, shift right = 2]
		\end{tikzcd}
		\end{center}
	\end{definition}	
	 Given a space $X$ there is an associated $\omega$-groupoid $\omega(X)$. Further, there are natural isomorphisms
	$$\pi_0(\omega(X)) \cong \pi_0(X) \quad \pi_1(\omega(X), x_0) \cong \pi_1(X,x) \quad \pi_n(\omega(X),x_0) \cong H_n(\widehat{X}_{x_0}, x_0)$$
	We can therefore rephrase the above strengthening as 
	\begin{theorem}
		The functor $\omega: \text{Top} \to \stinfty$ reflects weak equivalences between $CW$ complexes.
	\end{theorem}
	The functor $\omega$ can be interpreted purely in terms of another kind of higher groupoid, called an $\infty$-groupoid. A rough definition of $\infty$-groupoids is the following: 
	
	\begin{definition}[Pseudo-Definition] An $\infty$-groupoid $G$ is like an $\omega$-groupoid, e.g. it has sets of $n$-cells $G_n$, source and target maps, identities, and composition operations, except that whenever an axiom of an $\omega$-groupoid would assert that two $n$-cells $\alpha,\beta \in G_n$ are equal, instead they are required to be \textit{homotopic}. That is, we replace every axiom ``$\alpha = \beta$'' with the axiom $\exists H \in G_{n+1}$ such that $s(H) = \alpha$ and $t(H) = \beta$.''
	\end{definition}
	\begin{example}
		Let $G$ be an $\infty$-groupoid, and \begin{tikzcd}\bullet \ar[r, "f"] & \bullet \ar[r, "g"] & \bullet \ar[r, "h"] & \bullet \end{tikzcd} a diagram in $G$. That is, suppose we have $f,g,h \in G_1$, with compatible sources and targets as shown. 
		Then we have a composition operation $\circ_0^1$, and can apply it in two different orders to get the $1$-cells \begin{tikzcd} \bullet \ar[r, "(fg)h"] & \bullet \end{tikzcd} and \begin{tikzcd} \bullet \ar[r, "f(gh)"] & \bullet \end{tikzcd}. If $G$ were an $\omega$-groupoid, these cells would be equal. Instead, we can only conclude that there is some $2$-cell $A_{fgh}$ with source and target these two different paranthesizations, which we draw as:
		\begin{center}
		\begin{tikzcd}[sep = large]
			\bullet  \ar[rr, "(fg)h" {name=A}, bend left = 15, shift left = 2] \ar[rr, "f(gh)" {name=B}, swap, shift right = 2, bend right = 15]
				& 
				& \bullet
			\arrow["A_{fgh}"{inner sep=1}, Rightarrow, from=A, to=B]
		\end{tikzcd}
		\end{center}
	\end{example}
	The important of $\infty$-groupoids is that they have a homotopy theory that is equivalent to the homotopy theory of spaces; i.e. we have an equivalence of homotopy categories $\Ho(\text{Top}) \simeq \Ho(\infty\text{Grpd})$. 
	\par The $\omega$-groupoids are precisely the $\infty$-groupoids where all of the axioms hold not just up to homotopy, but rather ``strictly.'' 
	For this reason they are also referred to as \textit{strict $\infty$-groupoids}. 
	We therefore have a natural inclusion of categories $\stinfty \hookrightarrow \infty\text{Grpd}$. 
	This has a left adjoint $\st: \infty\text{Grpd} \to \stinfty$, called ``strictification,'' which we might think of as collapsing the homotopies witnessing the axioms (e.g. $A_{fgh}$ in the above example) into identities, thereby forcing all the axioms to hold strictly. 
	It is nonobvious, but it turns out that the functors $\omega$ and $\st$ are, up to homotopy, the same functor:
	there is a commutative triangle of functors
	\begin{center}
	\begin{tikzcd}
	\Ho(\text{Top}) \ar[rd, "\omega "] \ar[dd, "\sim " {rotate = 90, anchor = north} ] \\
		 & \Ho(\stinfty) \\
	\Ho(\infty \text{Grpd}) \ar[ru, "\st "]
	\end{tikzcd}
	\end{center}
	where the vertical arrow is an equivalence of categories. 
	Thus, from the higher categorical view, the generalized Homological Whitehead Theorem has a natural rephrasing as``the strictification functor reflects weak equivalences.''
	\par
	This higher-groupoid phrasing suggests a generalization: $\infty$-groupoids are the beginning of a hierarchy of homotopical objects called $(\infty,n)$ categories; they are precisely the $(\infty, 0)$-categories. $(\infty,n)$ categories are roughly thought of as being ``$(\omega,n)$-categories where all axioms hold only up to homotopy (invertible higher cells)'' and so we expect there to exist functors $\text{St}_n : (\infty,n)\text{Cat} \to (\omega,n)\text{Cat}$ for all $n$, left adjoint to inclusion functors. One might optimistically hope that these functors carry a similar amount of structure as the functor $\text{St}_0$ carries; in particular one might hope that they are conservative and comonadic. In this paper we will verify a small piece of this optimism by constructing $\text{St}_1$ and proving that it is conservative. Imprecisely, we shall prove that:
	\begin{theorem}
		There is a functor $\st_1: (\infty,1)\text{Cat} \to (\omega,1)\text{Cat}$, left adjoint to a natural inclusion functor. If $\C$ and $\D$ are $(\infty,1)$-categories that are either:
		\begin{itemize}
			\item $2$-truncated, i.e. are $(2,1)$-categories
			\item $2$-connected, i.e. are simply connected monoidal $\infty$-groupoids
		\end{itemize}		 
then a functor $F: \C \to \D$ of $(\infty,1)$ categories is a weak equivalence if and only if the strictification $\st_1(F): \st_1\C \to \st_1\D$ is a weak equivalence.
	\end{theorem} 
	(The precise statement is in Theorems \TODO{reference} and \TODO{reference}, and we justify a model-independence of the result in Theorem \TODO{reference}). The $2$-truncated case is part of a general pattern of weak $2$-categories being modelled via strict $2$-categories, whereas the proof of the $2$-connected case turns out to be a special case of the fact that Hochschild Homology detects weak equivalences between differential graded algebras.
	From this evidence, we conjecture that in general, $\st_1$ is conservative:
	\begin{conjecture}\label{conjecture:conservativity}
		The functor $\st_1: (\infty,1)\text{Cat} \to (\omega,1)\text{Cat}$ is conservative. 
	\end{conjecture}
	In light of the comparison with the $(\infty, 0)$ case, we regard this as a sort of ``Homological Whitehead Theorem for $(\infty,1)$ categories.''
	\\\\
	Our approach is based on splitting the functor $\st_1$ into two parts as follows: 
	$(\infty,1)$-categories are thought of as categories weakly enriched in $\infty-groupoids$.
	By strictifying every hom $\infty$-groupoid, we obtain a category enriched in $\omega$-groupoids. 
	However, this enrichment is weak, and so we must further strictify the \textit{enrichment} to obtain a category (strictly) enriched in $\omega$-groupoids, equivalently an $(\omega,1)$-category.
	If we think of strictification as a ``linearization,'' then this composite
	\begin{center}
	\begin{tikzcd}
		(\infty,1)\mathcal{Cat} \cong \infty\text{Grpd}\mathcal{Cat} \ar[r, "\st^{\text{local}}"] 
			& \stinfty^{\text{weak}}\mathcal{Cat} \ar[r, "\st^{\text{global}}"] & \stinfty^{\text{strict}}\mathcal{Cat} \cong (\omega,1)\mathcal{Cat}
	\end{tikzcd}
	\end{center}
	we think of as first ``locally linearizing'' followed by ``globally linearizing.''
	To prove that this composite functor is conservative, it suffices to prove that each individual functor is conservative.
	The first $\st^{\text{local}}$ is conservative because strictification of $\infty$-groupoids is conservative, and equivalences of $(\infty,1)$-categories are detected locally. 
	We do not succed in proving that $\st^{\text{global}}$ is conservative; a major difficulty is its nonlocal nature: the hom $\omega$-groupoid $\st^{\text{global}}(\C)[x,y]$ depends not just on every $\C[x,y]$ but on every composition map $\C[x,z] \otimes \C[z,y] \to \C[x,y]$. 
	\par Concretely, the adjunction of which $\st^{\text{global}}$ is the left adjoint is a change of enrichment adjunction along a lax monoidal adjunction: namely, the adjunction $\id: (\stinfty, \otimes) \leftrightarrows (\stinfty, \times): \id$ between the gray monoidal structure and the cartesian monoidal structure on $\stinfty$. 
	This adjunction is not strong monoidal, nor even Quillen monoidal, the homotopical analog.
	We therefore have few general tools with which to analyze the change of enrichment adjunction.
	In future work, we hope to complete the proof of conjecture \ref{conjecture:conservativity} via analyzing a similar adjunction involving categories enriched in chain complexes over a varying groupoid of operators.
	% A major contrast of the development of the study of $\infty$-groupoids to the study of the development of $(\infty,1)$ categories is that many of the $\infty$-groupoids of interest to mathematicians come prepackaged as a collection of cells; the utility of homology is largely how easy it is to compute from these cells. On the other hand, the $(\infty,1)$ categories most of interest to mathematicians tend to arise as ``larger'' objects, e.g. as homotopy theories.\footnote{Similarly, many groups come given as generators and relations, whereas few categories of immediate independent interest are easily ``presented'' in such a way.} However, just as homology is an indispensable tool even for the study of spaces whose homology is not necessarily immediately computable, so too we hope that $(\omega,1)$ categories and their comparatively simple homotopy theory will allow some new progress in the study of $(\infty,1)$ categories.
\subsection{Conventions}
	We shall be relatively concrete about our models, working mainly with particular model categories.
	By $(\omega,n)$-category we mean the strictest possible notion (which we recall in section 1). 
	There are (as is typical) obvious set theoretic issues as we deal with ``categories of categories''; we do not deal with these but refer the interested reader to \TODO{reference something} for a summary of how this would be done. 
	\\\\
	For $\V$ a symmetric monoidal category with unit $\mathbb{I}$ and initial object $\emptyset$, we shall denote by $\sus[-]: \V \to \V-\mathcal{Cat}$ the functor which takes an object $V$ to the $\V$-category with two objects $0$ and $1$, with
	$$\sus(V)[0,0] = \sus(V)[1,1] = \mathbb{I} \quad \quad \sus(V)[0,1] = V \quad \quad \sus(V)[1,0] = \emptyset$$
	\\\\
	In all of the model categories we will consider, we will write composition in the standard order, e.g. if we have morphisms \begin{tikzcd}  X \ar[r, "f"] & Y \ar[r, "g"] & Z \end{tikzcd} we write the composition $gf$. 
	However, for other categorical compositions we will generally write composition the opposite way, e.g. if we have morphisms \begin{tikzcd}  p \ar[r, "\ell "] & q \ar[r, "k"] & r \end{tikzcd} of a groupoid, then we will write their composition as $\ell k$. 
\subsection{Organization of this paper}
	In section 1 we recall the basic definitions of $\omega$ groupoids and $(\omega,1)$ categories, as well as at the Gray tensor product, which will feature heavily in the proof of the main theorem. In section 2 we recall the model structures on our four categories of interest, define the relevant quillen adjunctions, and finally give the proof of the main theorem, which can be outlined imprecisely as follows: $(\infty,1)$-categories are equivalently categories weakly enriched in $\infty-groupoids$.
	By strictifying every hom $\infty$-groupoid, we obtain a category enriched in $\omega$-groupoids. 
	However, this enrichment is weak, and so we must further strictify the \textit{enrichment} to obtain a category (strictly) enriched in $\omega$-groupoids, equivalently an $(\omega,1)$-category.
	If we think of strictification as a ``linearization,'' then this composite
	\begin{center}
	\begin{tikzcd}
		(\infty,1)\mathcal{Cat} \cong \infty\text{Grpd}-\mathcal{Cat} \ar[r, "\st^{\text{local}}"] 
			& \stinfty^{\text{weak}}-\mathcal{Cat} \ar[r, "\st^{\text{global}}"] & \stinfty^{\text{strict}}-\mathcal{Cat} \cong (\omega,1)\mathcal{Cat}
	\end{tikzcd}
	\end{center}
	we think of as first ``locally linearizing'' followed by ``globally linearizing.''
	To prove that this composite functor is conservative, it suffices to prove that each individual functor is conservative. 
	The first $\st^{\text{local}}$ is conservative because strictification of $\infty$-groupoids is conservative, and equivalences of $(\infty,1)$-categories are detected locally. 
	Proving that $\st^{\text{global}}$ is conservative requires more original work, requiring a nonabelian and multi-object generalization of the fact that a map between free $\text{dg}$-algebras is a quasi-equivalence if it induces an equivalence on the prime elements.
	We carry this analysis out via means of a spectral sequence, the details of which we relegate to section 3 (after the main proof). Finally, in section 4 we discuss some potential applications and relations with existing work, as well as sketch a proof strategy for constructing and analyzing functors $\st_n: (\infty,n)\mathcal{Cat} \to (\omega,n)\mathcal{Cat}$. 
%	for $\C$ an $(\infty,1)$-category, given concretely as an $\sset$-category, we can apply $\st_0$ to each of its hom spaces to obtain an $\stinfty$-category.
%	Since this preserves the homotopy category, and $\st_0$ reflects weak equivalences, we can deduce that this operation reflects weak equivalences (since the weak equivalences between $\sset$-categories are those functors which are equivalences on the homotopy categories and on the hom-spaces).
%	However, since is Quillen monoidal with respect to the gray tensor product (rather than the cartesian product) of $\omega$-groupoids, this resulting $\stinfty$-category is enriched with respect to the gray tensor product, not the cartesian product, so it is not an $(\omega,1)$-category.
%	To obtain an $(\omega,1)$ category we must apply a somewhat obtuse functor which has the behaviour of killing off nontrivial tensor products, and which we therefore think of as a sort of ``global linearizing'' (as opposed to the first functor, which is a sort of ``local linearizing''). 
%	Proving that this functor reflects weak equivalences is nontrivial; we do so by means of a spectral sequence, the details of which we relegate to section 3. 
%	Finally, in section 4 we discuss some potential applications and relations with existing work. 
\section{Background: $(\omega,n)$ categories and crossed complexes}
	\begin{definition}\label{dfn:omega-cats} 
		An \emph{$\omega$-category} $C$ is a sequence of sets
		\begin{center}
		\begin{tikzcd}[sep = huge]
		C_0 & C_1 \ar[l, "s", shift left = 2] \ar[l,"t" swap, shift right = 2] & C_2 \ar[l, "s", shift left = 2] \ar[l,"t" swap, shift right = 2]   & \cdots \ar[l, "s", shift left = 2] \ar[l,"t" swap, shift right = 2] 
		\end{tikzcd}
		\end{center}
		such that each diagram
		\begin{center}
		\begin{tikzcd}[sep = huge]
		C_i & C_{i+k} \ar[l, "s^k", shift left = 2] \ar[l,"t^k" swap, shift right = 2] 
		\end{tikzcd}
		\end{center}
		is equipped with the structure of a category, and such that these are compatible in the sense that 
		\begin{center}
		\begin{tikzcd}[sep = huge]
		C_i & C_{i+k} \ar[l, "s^k", shift left = 2] \ar[l,"t^k" swap, shift right = 2]  & C_{i+k+j} \ar[l, "s^j", shift left = 2] \ar[l,"t^j" swap, shift right = 2]  & 
		\end{tikzcd}
		\end{center}
		is a ``strict $2$-category.'' This is equivalent to imposing that  
		$$(\alpha \circ_j^k \beta) \circ_i^k(\gamma \circ_j^k \eta) = (\alpha \circ_i^k \beta) \circ_j^k(\gamma \circ_i^k \eta)$$
		whenever these are defined, where $\circ_a^b$ denotes the categorical composition operation of 
		\begin{center}
		\begin{tikzcd}[sep = huge]
		C_a & C_b \ar[l, "s^{b-a}", shift left = 2] \ar[l,"t^{b-a}" swap, shift right = 2] 
		\end{tikzcd}
		\end{center} 
		This axiom is referred to as ``interchange.'' A map between $\omega$-categories is a map of diagrams which preserves all the categorical structure. The resulting category we notate as $\omegacat$.
	\end{definition}
	\begin{definition}\label{dfn:omega-n-cats}
		An \emph{$(\omega,n)$-category} $C$ is an $\omega$-category such that for all $i \ge n$ and $k > 0$, the categories
		\begin{center}
		\begin{tikzcd}[sep = huge]
		C_i & C_{i+k} \ar[l, "s^k", shift left = 2] \ar[l,"t^k" swap, shift right = 2] 
		\end{tikzcd}
		\end{center}
		are groupoids. The resulting full subcategory of $\omegacat$ we notate as $\omegancat{n}$. In the special case $n = 0$, we refer to $(\omega,n)$-categories as $\omega$-groupoids, and denote the category by $\stinfty$.
	\end{definition}
	$\omegancat{n}$ possess small limits, and consequentially we can define enriched categories over it:
	\begin{definition}
		We denote by $\cartesiancat{n}$ the category whose objects are categories enriched in $\omegancat{n}$ with respect to the cartesian monoidal structure, and morphisms functors between such categories. 
	\end{definition}
	A key feature is that enriching in $(\omega,n)$-categories ``raises the categorical level by $1$'':
	\begin{theorem}\label{thm:enriched-strict-equivalence}
		There is an equivalence of categories
			$$\omegannerve{n}:\cartesiancat{n} \leftrightarrows (\omega,n+1)\text{Cat}: \omeganrigidification{n+1}$$ 
	\end{theorem}
	\TODO{the notation should be mathfrak, but there's some font error. ugh.}
	\begin{proof}
		As the result is classical, we define just the functor $\omegannerve{n}$ for the point of illustration: for $\C \in \cartesiancat{n}$, we define $\omegannerve{n}(C)$ by:
		$$\omegannerve{n}(C)_0 = \ob(C) \quad \text{ and } \quad \omegannerve{n}(C)_k= \coprod_{x,y \in \ob(C)} C[x,y]_{k-1}\text{ for } k > 0$$ 
		The source and target maps are given ``locally'' (i.e. by the source and target maps of $\C[x,y]$) for $k > 1$, and for $k = 1$ $s(\alpha) = x$ where $x$ is the unique object of $C$ with $\alpha \in C[x,y]$ (and likewise for $t$. 
		Similarly, the composition operations $\circ_i^k$ are given ``locally'' if $i > 0$; i.e. if $i > 0$ then for $\alpha \circ_i^k \beta$ to be defined we must have that $\alpha, \beta \in \C[x,y]$ for some objects $x$ and $y$ and so we can define 
		$$\alpha \circ_i^k \beta = \alpha \tilde{\circ}_{i-1}^{k-1} \beta$$
		Where $\tilde{\circ}$ denotes the composition operation in $\C[x,y]$. 
		It is not hard to see that interchange is satisfied for $i < j < k$ when $i > 0$; when $i > 0$ it follows from the composition maps in $C$ being maps of $(\omega,n)$-categories. 
		The fact that the appropriate category structures are groupoidal (so that $\omegannerve{n}(C)$ is not just an $\omega$-category, but an $(\omega,n+1)$-category) follows directly from the fact that the corresponding category structures in $C$ (but shifted down one indexing level) are groupoidal.
	\end{proof}
	For the results of this paper, we will only really need to know that $(\omega,1)\text{Cat}$ is equivalent to $\cartesiancat{0}$. 
	Furthermore, we will not actually work directly with $\omega$-groupoids: 
	$\omega$-groupoids admit an alternative description, due to \TODO{reference original} as a sort of nonabelian chain complex called a \textit{crossed complex}
	\begin{definition}[\cite{Brown_Higgins_Sivera_2011}, Definition 7.1.9]
	A \textbf{crossed complex} $C$ is a sequence of sets 
	\begin{center}
	\begin{tikzcd}
		C_0 & C_1 \ar[l, "s", shift right = 2, swap] \ar[l, "t", shift left = 2] & C_2 \ar[l, "\delta_2"] &  C_3 \ar[l,"\delta_3"] & \cdots \ar[l]
	\end{tikzcd}
	\end{center}
	Such that:
	\begin{enumerate}
		\item The diagram
	\begin{center}
	\begin{tikzcd}
		C_0 & C_1 \ar[l, "s", shift right = 2, swap] \ar[l, "t", shift left = 2]
	\end{tikzcd}
	\end{center}
	forms a groupoid, with which we will sometimes abuse notation by referring to simply as $C_1$.
	\item Each $C_i$ for $i \ge 2$ is a skeletal groupoid on object set $C_0$ over the groupoid $C_1$: that is, a family of groups of the form 
	$$C_i = \coprod_{c \in C_0} C_i(c)$$
	where each $C_i(c)$ is a group, equipped with morphisms
	$$
	\phi_\ell: C_i(s(\ell)) \to C_i(t(\ell))
	$$
	for each $\ell \in C_1$, satisfying that for composable $\ell$ and $p$ in $C_1$, 
	$$
	\phi_{\ell \circ p} = \phi_\ell \circ \phi_p
	$$
	and that $\phi_{\id_x} = \id_{C_i(x)}$. Further, each $C_i(c)$ is abelian for $i>2$. From now on we shall generally suppress the $ c \in C_0$ from our notation when our meaning is clear, saying for example ``$C_i$ is abelian for $i > 2$.''
	\item For $i > 2$, the maps $\delta_i$ are families of maps of groups $\delta_i : C_i \to C_{i-1}$, satisfying $\delta_{i-1} \circ \delta_i = 0$. For the case $i=2$, we have that $s\circ \delta_2 = t\circ \delta_2$.
	\item $\delta_2$ is a family of maps of groups $\delta_2(c) : C_2(c) \to \Aut(c)$, where by $\Aut(c)$ we mean the automorphism group of $c$ in the groupoid $C_1$.
	\item The action of $C_1$ on $C_i$ is compatible with the $\delta_i$ in the sense that for $i > 2$, $\ell \in C_1$, and $a \in C_i(x)$
	$$\phi_\ell \circ \delta_i = \delta_{i-1} \circ \phi_\ell$$
	\item For any $a \in C_2$, $\delta_2(a)$ acts by conjugation by $a$ on $C_2$ and trivially on $C_i$ for $i > 2$. 
	Note that imposing that the image of $C_2$ acts trivially on $C_i$ for $i > 2$ is equivalent to asking that these $C_i$ are in fact modules over the cokernel of $C_2 \to C_1$.
	\end{enumerate}
	\end{definition}
	\begin{theorem}[\TODO{reference, notation}]\label{thm:crossed-omega-equivalence}
		There is an equivalence of categories
		$$ \puttogether : \crcom \rightleftarrows \stinfty : \takeapart$$
	\end{theorem}
	One should think of a crossed complex as a gadget which encodes a sort of ``basis'' for the cells of the associated $\omega$Gpd; the groups $C_n(x)$ are roughly to be thought of as ``unbased $n$-cells'' in the following sense: for $G \in \stinfty$, the $n$-cells $\alpha \in G_n$, one can whisker (compose with identity cells) $\alpha$ with the lower dimensional cells $s(\alpha)^{-1}, s^2(\alpha)^{-1},...,s^{n-1}(\alpha)^{-1}$ to obtain an $n$-cell $\tilde{\alpha}$ such that $s(\tilde{\alpha})$ is equal to $\id^{n-1}(p)$ for $p$ the $0$-cell $s^n(\alpha)$. 
	The $n$ cells of this form are a group under the composition operation in 
	\begin{center}
		\begin{tikzcd}[sep = huge]
		C_{n-k} & C_{n} \ar[l, "s^k", shift left = 2] \ar[l,"t^k" swap, shift right = 2] 
		\end{tikzcd}
		\end{center}
if $n \ge 2$, $ 2 \le k \le n$; by the Eckmann-Hilton argument these groups are abelian if $n \ge 3$.
We think of these $n$-cells $\tilde{\alpha}$ as being ``unbased $n$-cell data''; given $s(\alpha)$ and $\tilde{\alpha}$ we can recover $\alpha$ via whiskering; moreover for any two composable $\alpha$ and $\beta$ in $G_n$ we can recover the composition $\alpha \circ_{n-1} \beta$ via $\tilde{\alpha} + \tilde{\beta}$ and $s(\alpha)$. 
	The effect of the functor $\takeapart$ is remember only the cells $\tilde{\alpha}$ and assemble them into a crossed complex, and the effect of $\puttogether$ is to take the cells $\tilde{\alpha}$ and recursively piece them together to recover the original $n$-cells.
	\\
	\indent The reason that we are concerned with $\crcom$ is that its chain-complex-like nature makes it often easier to work with directly than $\stinfty$; in particular we have by the above an equivalence of categories $\omegancat{1} \cong \cartcrossedcat $, and for the remainder of the paper we will use the later as our model of choice for $(\omega,1)$-categories.
	\begin{theorem}\label{thm:crossed-complex-reflects-weak-equivalences}
		The fundamental crossed complex functor $\st_0: \sset \to \crcom$ reflects weak equivalences.
	\end{theorem}
\subsection{Monoidal structures on Crossed Complexes} 
The category of crossed complexes has two monoidal structures that are of interest to us in this paper: the first is the cartesian monoidal structure.
\begin{theorem}
	Let $C,D \in \crcom$. Their cartesian product $C \times D$ is the crossed complex
	\begin{center}
	\begin{tikzcd}
		C_0 \times D_0 & C_1 \times D_1 \ar[l, "s", shift right = 2, swap] \ar[l, "t", shift left = 2] & C_2 \times D_2 \ar[l, "\delta_2"] &  C_3 \times D_3 \ar[l,"\delta_3"] & \cdots \ar[l]
	\end{tikzcd}
	\end{center}
	where $C_0 \times D_0$ is the product of sets and $C_i \times D_i$ is the (morphism set of the) cartesian product of groupoids. The boudnary maps and the action of $C_1 \times D_1$ are given component wise, and the natural maps $C \times D \to C$ and $C \times D \to D$ are given by levelwise projections.
\end{theorem}
The proof is routine. 
\par The second monoidal structure on $\crcom$ that we are interested in we shall simply call the \textit{tensor product} and notate by $\otimes$. This product corresponds to the gray tensor product of $\omega$-groupoids along the equivalence of categories $\stinfty \cong \crcom$. An abstract definition is given in \TODO{reference}, we give here a relatively concrete generators-and-relations type definition:
\begin{definition}
	For $C,D \in \crcom$, their tensor product $C \otimes D$ is presented as follows:\footnote{see the tonks eilenberg zilber paper} 
	it has generators given by $c_m \otimes d_n \in (C \otimes D))_{m+n}$ for $c_m \in C_m$	and $d_n \in D_n$. The source and (target maps, when relevant) are given componentwise. 
	The relations are given as follows:
	$$
	(c_m \otimes d_n)^{a_i \otimes b_j} = 
	\begin{cases}
		c_m^{a_i} \otimes d_n & \text{if }m \ge 2\text{ and } b_j = sd_n \\
		c_m \otimes d_n^{b_j} & \text{if } n \ge 2 \text{ and } a_i = sc_m
	\end{cases}
	$$
	$$
	(c_mc'_m \otimes d_n) = 
	\begin{cases}
		c_m \otimes d_n \cdot c'_m\otimes d_n & \text{if }m = 0 \text{ or } n \ge 2 \\
		c'_m \otimes d_n \cdot (c_m \otimes d_n)^{c'_m \otimes sd_n} & \text{if } n = 1 \text{ and } m \ge 1
	\end{cases}
	$$
	Symmetrically, we have
	$$
	(c_m \otimes d_nd'_n) = 
	\begin{cases}
		c_m \otimes d_n \cdot c_m\otimes d'_n & \text{if } n = 0 \text{ or } m \ge 2 \\
		(c_m \otimes d_n)^{sc_m \otimes d_n'} \cdot (c_m \otimes d'_n) & \text{if } n = 1 \text{ and } m \ge 1
	\end{cases}
	$$
	The boundary maps are given by 
	$$\del (c_m \otimes d_n) = \del(c_m) \otimes d_n \cdot c_m \otimes \del(d_n)$$
	Except in the cases where the degree(s) are $1$. If $m = 1$, replace the first term with
	$$(sc_1 \otimes d_n)^{-1} \otimes (tc_1 \otimes d_n)^{c_1 \otimes sd_n}$$
	and similarly for the case $n = 1$.
	If $m = n = 1$, then in particular 
	$$\del(c_1 \otimes d_1) = (sc_1 \otimes d_1)^{-1} \cdot (c_1 \otimes td_1)^{-1} \cdot (tc_1 \otimes d_1) \cdot (c_1 \otimes sd_1) $$
\end{definition}
The unit for $\otimes$ is given by the ``one-point crossed complex'' $\bullet$ in which the underlying groupoid is the groupoid with one object and no nonidentity morpisms, and all the higher groups are trivial. 
Since this is also the terminal object, for crossed complexes $C$ and $D$ we have maps $C \otimes D \to C \otimes \bullet \cong C$ and $C \otimes D \to \bullet \otimes D \cong D$; 
these induce a map $C \otimes D \to C \times D$. 
\begin{example}\label{example:tensor-v-product-intervals}
	let $I$ be the crossed complex with $C_1$ the contractible groupoid \begin{tikzcd} 0 \ar[r, "\ell"] & 1\end{tikzcd}, and consider the tensor product $I \otimes I$. By the description above, the underlying groupoid of $I \otimes I$ has four generators, which form a \textit{non-commuting} square
	\begin{center}
	\begin{tikzcd}
		0\otimes 0  \ar[r, "\ell \otimes 0"] \ar[d, "0 \otimes \ell"]&  1 \otimes 0 \ar[d, "1 \otimes \ell"] \\
		0 \otimes 1 \ar[r, "\ell \otimes 1"] & 1 \otimes 1
	\end{tikzcd}
	\end{center}
	There is one generator in dimension 2, the tensor $\ell \otimes \ell$, and the boundary is exactly the square above (that is, the composition of the morphisms in the square given by starting at $0 \otimes 0$ and then going once counterclockwise around the square). 
	Therefore, this is a ``weakly commuting'' square, with $\ell \otimes \ell$ witnessing the weak commutativity. 
	By contrast, the crossed complex $I \times I$ is the ``strictly commuting'' square, whose underlying groupoid is the commuting square
	\begin{center}
	\begin{tikzcd}[sep = large]
		{(0{,} 0)}  \ar[r, "(\ell{,} {\id_0})"] \ar[d, "({\id_0}{,}\ell)"] \ar[rd, "({\ell}{,} \ell)"] 
			&  {(1{,}0)} \ar[d, "({\id_1}{,} \ell)"] \\
		{(0{,} 1)} \ar[r, "(\ell{,} {\id_1})"] 
			& {(1{,} 1)}
	\end{tikzcd}
	\end{center}
	The natural map $I \otimes I \to I \times I$ is given by collapsing the generating $2$-cell $\ell \otimes \ell$.
\end{example}
This example demonstrates the general behaviour of the map $C \otimes D \to C \times D$, which can be described as collapsing all of the tensors $c_m \otimes d_n$ where both $m$ and $n$ are nonzero. 
It sends a tensor $c_m \otimes d_0$ to the element $(c_m, e)$ where $e$ is the identity element at $s(d_n)$, and similarly for tensors $c_0 \otimes d_n$. 
This description is fairly immediate from the definition of the map. 
\section{The Main Constructions: Model Structures and Quillen Adjunctions}
	The object of this paper is to show that the composition of left adjoints in the diagram
	\begin{center}
	\begin{tikzcd}[sep = large]
		\ssetcat \ar[r, bend left = 20, "\leftone "] & \tensorcrossedcat \ar[l, bend left = 20, "\rightone "] \ar[r, bend left = 20, "\lefttwo "] & \cartcrossedcat \ar[l, bend left = 20, "\righttwo "]
	\end{tikzcd}
	\end{center}
	reflects weak equivalences between cofibrant objects. 
	In this section we specify what the model structures are on each category, and describe how the functors are induced. 
\subsection{The Bergner model structure on $\ssetcat$}
	Our model of choice for $(\infty,1)$ categories will be the Bergner model structure on $\ssetcat$. We recall here the definitions: \\
	\begin{theorem}[\TODO{reference Bergner}]
		There is a model category structure on $\ssetcat$ where a morphism $f: \C \to \D$ is:
		\begin{enumerate}
			\item A weak equivalence if it induces an equivalence of categories $\pi_0 \C \to \pi_0 \D$, as well as weak equivalences of simplicial sets $\C[x,y] \to \D[fx,fy]$ for all $x,y \in \ob \C$.
			\item A fibration if it induces an isofibration of homotopy categories $\Ho\C \to \Ho \D$, as well as fibrations of simplicial sets $\C[x,y] \to \D[fx,fy]$ for all $x,y \in \ob \C$.
			\item A cofibration if it has the left lifting property with respect to the acylic fibrations.
		\end{enumerate}
	\end{theorem}
	\begin{definition}
		Let $I'$ be the set of functors of $\sset$-categories of the form $\sus[\del \Delta^n \to \Delta^n]$. A \emph{simplicial computad} is an $I'$-cellular object in $\ssetcat$, and a \textit{relative simplicial computad} is an $I'$-cellular morphism in $\ssetcat$.
	\end{definition}
	In other words, a simplicial computad is an $\sset$-category which is given by freely adjoining simplices to individual hom-spaces. 
\subsection{The model structure on $\tensorcrossedcat$.}
	In \TODO{reference} the authors show that $\omegancat{n}$ is a monoidal model category (symmetric for $n = 0$ and satisfies the monoid axiom of \TODO{reference}. In particular, since the symmetric monoidal structure is equivalent to the tensor product $\otimes$ of crossed complexes under the equivalence of categories between $\stinfty$ and $\crcom$ (see \TODO{reference}), $\crcom$ is a monoidal model category satisfying the monoid axiom. 
	Since the unit for $\otimes$ is cofibrant, it furthermore satisfies the weak unit axiom of \TODO{reference}.
	Finally, we note that $\crcom$ is combinatorial, and closed monoidal, 
	so overall it is a combinatorial closed symmetric monoidal model category satisfying the monoid axiom.s
	It then follows by \TODO{reference Muro}:
	\begin{theorem}
		There is a model category structure on $\tensorcrossedcat$ in which a map $f: \C \to \D$ is:
		\begin{itemize}
			\item A weak equivalence if each of the maps $\C[x,y] \to \D[fx,fy]$ is a weak equivalence in $\crcom$, and also the induced map $\Ho(f): \Ho(\C) \to \Ho(\D)$ is an equivalence of categories
			\item An acyclic fibration if it is surjective on objects and each of the maps $\C[x,y] \to \D[fx,fy]$ is an acyclic fibration in $\crcom$
			\item A cofibration if it has the left lifting property with respect to the acyclic fibrations
		\end{itemize}
	\end{theorem}
	This is phrased slightly differently then in \TODO{reference Muro}; in that paper the notion of weak equivalence is one in which $\Ho(f) : \Ho(\C) \to Ho(\D)$ is essentially surjective. 
	However, if the maps $\C[x,y] \to \D[fx,fy]$ are weak equivalences then in particular they are $\pi_0$-isomorphisms; hence $\Ho(f)$ is fully faithful, and so an equivalence if and only if it is essentially surjective.

\subsection{\TODO{clean up and clarify} The model structure on $\cartcrossedcat$}
	\TODO{in some ways this is the simplest, but it is really the trickiest - it's not immediately obvious to me whether or not $\crcom$ is even a cartesian model category; if it is then it should be the same model structure on enriched categories as the folk model structure but that is also not immediately obvious. must come back to this to clarify. Worst case I have to go through the arguments by hand and say ``well, we aren't in this situation, but we still have the adjoint and it is a quillen pair because blah''}
	\\\\
	In general, $\omegancat{n}$ has a model structure due to \TODO{reference}. For our purposes, we only need to consider the model structure on $\omegancat{1}$. It turns out the model structure in  ----- coincides with the one induced by the Dwyer-Kan model structure and the equivalence of categories with $\cartcrossedcat$. Recall
	\begin{definition}
	Let $C$ be an $(\omega,1)$-category, $n \in \mathbb{N}$.
		\begin{itemize}
			\item We say that an $1$-cell $x \in C_n$ is \emph{equivalent} to $y \in C_n$ if there is a reversible $(n+1)$-cell $u: x \to y$.
			\item We say that an $(n+1)$-cell $u \in C_{n+1}$ is \emph{reversible} if there is an $(n+1)$ cell $\overline{u} \in C_{n+1}$ such that $u * \overline{u}$ and $\overline{u} * u$ are each equivalent to identity $(n+1)$-cells.
		\end{itemize}
	\end{definition}
	\begin{definition}
		Let $\mathcal{W}$ be the class of morphisms $f: C \to D$ in $\omegancat{1}$ such that:
		\begin{enumerate}
			\item For each $0$-cell $y$ in $D$, there is a $0$-cell $x$ in $X$ such that $fx$ is $\omega$-equivalent to $y$.
			\item For each pair of parallel $n$-cells $x$ and $x'$ in $C_n$, and each $(n+1)$-cell $v:fx \to fx'$ in $D$, there is an $(n+1)$-cell $u:x \to x'$ such that $f u$ is equivalent to $v$.
		\end{enumerate}
	\end{definition}
	\begin{definition}
		Let $C$ be an $(\omega,1)$-category. The \textit{homotopy category} of $C$ is the $1$-category with objects those of $C$, and morphisms the $1$-cells of $C$ modulo the relation of $\omega$ equivalence. Note that 
	\end{definition}
	\begin{lemma}: 
		A functor $f: C \to D$ of $(\omega,1)$ categories is an equivalence if and only if:
		\begin{itemize}
			\item $\Ho(f)$ is essentially surjective, and
			\item for each $x,y \in C_0$, $f: C[x,y] \to D[x,y]$ is a weak equivalence of $\omega$-groupoids (in the sense of being a weak equivalence of $\omega$-categories).
		\end{itemize}
	\end{lemma}
	\begin{proof}
		Do I include this? Seems too obvious.
	\end{proof}
	\begin{theorem}
		There is a model structure on $(\omega,1)-\text{Cat}$ in which:
		\begin{itemize}
			\item A functor $F: \C \to \D$ is a weak equivalence precisely if it induces an essential surjection of homotopy categories $\text{Ho}(F) : \text{Ho}(\C) \to \text{Ho}(\D)$ and for every $x,y \in \text{Ob}(\C)$ it induces a weak equivalence of $(\omega,0)$-categories $\C[x,y] \to \D[Fx,Fy]$.
			\item A functor $F \C \to \D$ is a trivial fibration precisely if it is surjective on objects and for every $x,y \in \text{Ob}(\C)$ it induces a trivial fibration of $(\omega,0)$-categories $\C[x,y] \to \D[Fx,Fy]$. 
		\end{itemize}			 
	\end{theorem}
	\begin{proof}
		See \TODO{reference}
	\end{proof}
	This immediately gives us the following model structure:
	\begin{theorem}
		There is a model structure on $\cartcrossedcat$ in which
		\begin{itemize}
			\item A functor $F: \C \to \D$ is a weak equivalence precisely if it induces an essential surjection of homotopy categories $\text{Ho}(F) : \text{Ho}(\C) \to \text{Ho}(\D)$ and for every $x,y \in \text{Ob}(\C)$ it induces a weak equivalence of crossed complexes $\C[x,y] \to \D[Fx,Fy]$.
			\item A functor $F \C \to \D$ is a trivial fibration precisely if it is surjective on objects and for every $x,y \in \text{Ob}(\C)$ it induces a trivial fibration of crossed complexes $\C[x,y] \to \D[Fx,Fy]$. 
		\end{itemize}
	\end{theorem}
\subsection{The model structure on $\grpdcat$}
	 The category $\grpd$ of (ordinary) groupoids is cartesian closed, with the mapping groupoid $\grpd [G,H]$ being the category of functors $G \to H$ (which is a groupoid). Furthermore, it is a closed monoidal model category with respect to the cartesian monoidal structure, so we have a model structure on $\grpdcat$. 
	 \begin{theorem}
	 	There is a model structure on $\grpdcat$, the category of categories enriched in $\grpd$ with the cartesian monoidal structure, in which a morphism $F: \C \to \D$ is:
	 	\begin{itemize}
	 		\item A weak equivalence if it induces an equivalence of categories $\ho(F): \ho(\C) \to \ho(\D)$, and
	 		\item A fibration if it is surjective on objects and a local fibration
	 	\end{itemize}
	 \end{theorem}

\section{The adjoint functors}
	In this section we define all of the quillen adjoint pairs that we discuss towards proving our main theorem. First, we briefly and somewhat informally recall some enriched category theory: recall that if $\V$ in which $L$ presand $\W$ are monoidal categories and $L: \V \leftrightarrows \W: R$ an adjunction with $L$ strong monoidal (meaning $L$ preserves the monoidal product up to natural isomorphism), then they induce an adjunction $L: \V-\mathcal{Cat} \leftrightarrows  \W-\mathcal{Cat}$ between categories of enriched categories, called the \textit{change of enrichment} or sometimes \textit{change of base}. These are given as the identity on object sets and by applying $L$ or $R$ locally to the hom-objects. However, often we are presented with an adjunction in which $L$ does not preserve the monoidal product, and would still like to define an adjunction $\V-\mathcal{Cat} \leftrightarrows \W-\mathcal{Cat}$. A classical theorem tells us that we can do this provided $R$ is lax monoidal:
	\begin{definition}
		A \emph{lax monoidal} functor $F: \V \to \W$ between monoidal categories is a functor equipped with a map $\mathbb{I}_\W \to F(\mathbb{I}_\V)$ and a natural transformation $F(-) \otimes_\W F(-) \to F(- \otimes_\V -)$, satisfying associativity and unitality conditions.
	\end{definition}
	
	\begin{theorem}[\TODO{reference}] \label{thm:lax-monoidal-induces-enriched}
		Let $\V$ and $\W$ be monoidal categories, and $L : \V \leftrightarrows \W: R$ an adjunction with $R$ lax monoidal. Then there is an induced adjunction 
		$$L^{\text{cat}}: \V-\mathcal{Cat} \leftrightarrows \W-\mathcal{Cat}: R$$
		In which both adjoints are identity on objects, and for $\C \in \W-\mathcal{Cat}$ and $x,y \in \ob \C$, $R(\C)[x,y] = R(\C[x,y])$, with composition maps given on $R(\C)$ defined using the lax monoidal structure of $R$. If $L$ is strong monoidal, then $L^{\text{cat}}$ is also defined by applying $L$ locally.
	\end{theorem}
	In general if $L$ is not strong monoidal, the left adjoint $L^{\text{cat}}$ is not given by applying $L$ locally. This makes it both more interesting but also much harder to study. However, \TODO{muro} shows that if $\V$ and $\W$ are monoidal model categories and $L$ is Quillen monoidal (a homotopical generalization of strong monoidal), then the derived functor of $L^{\text{cat}}$ can be calculated by applying $L$ locally. 
	\par Our quillen pairs will be obtained by applying the previous theorem to the following commutative (up to natural isomorphism) diagram of adjoint quillen pairs between monoidal model categories:
	\begin{center}
	\begin{tikzcd}[sep = huge]
		\sset^\times \arrow[r, bend left = 5, shift left = 1, "\st_0 "] \arrow[rd, bend left = 10, shorten = 5, "\Pi_1 "] 
			& \crcom^\otimes \arrow[r, bend left = 5, shift left = 1, "\Id "] \ar[l, bend left = 5, "U_0 "] \ar[d, bend left = 10, shift left = 2, "\Pi_1 "]
			& \crcom^\times \ar[l, bend left = 5, "\Id "] \ar[ld, bend left = 10, shorten = 5, "\Pi_1 "] 
		\\
			& \grpd^\times \ar[lu, bend left = 10, shorten = 5, "N"] \ar[u, bend left = 10, "\iota "] \ar[ru, bend left = 10, shorten = 5, "\iota "]
	\end{tikzcd}
	\end{center}
	In this diagram, each right adjoint is lax monoidal (and in fact, the vertical and diagonal left adjoints are strong monoidal). By Theorem \ref{thm:lax-monoidal-induces-enriched}, we have an induced diagram of adjoints, which we name as in the below diagram:
	\begin{center}
	\begin{equation}\label{diagram:enriched-functors}
	\begin{tikzcd}[sep = huge]
		\ssetcat \arrow[r, bend left = 5, shift left = 1, "\leftone "] \arrow[rd, bend left = 10, shorten = 5, "\hotwo "] 
			& \tensorcrossedcat \arrow[r, bend left = 5, shift left = 1, "\lefttwo "] \ar[l, bend left = 5, "\rightone "] \ar[d, bend left = 10, shift left = 2, "\hotwo "]
			& \cartcrossedcat \ar[l, bend left = 5, "\righttwo "] \ar[ld, bend left = 10, shorten = 5, "\hotwo "] 
		\\
			& \grpdcat \ar[lu, bend left = 10, shorten = 5, "N"] \ar[u, bend left = 10, "\iota "] \ar[ru, bend left = 10, shorten = 5, "\iota "]
	\end{tikzcd}
	\end{equation}
	\end{center} 
	Just as we abuse notation by using $\Pi_1$ to denote the fundamental groupoid for both simplicial sets and crossed complexes, we abuse notation by using $\hotwo$ to denote the three induced functors; in each case we think of it as the functor which takes an enriched category to its homotopy $(2,1)$-category. We also use $\iota$ and $N$ to describe both the right adjoints out of $\grpd$ and out of $\grpdcat$. 
	We will describe each of these in turn and prove they are quillen pairs. 
\subsection{The adjunction between $\ssetcat$ and $\tensorcrossedcat$.}
	We describe the ``local strictification'' adjunction 
	$$
		\leftone: \sset-\text{Cat} \rightleftarrows \tensorcrossedcat: \rightone
	$$
	The adjunction is induced using Theorem \ref{thm:lax-monoidal-induces-enriched} from the adjunction $\text{St}_0: \sset \rightleftarrows \crcom :  U_0$.
	\begin{theorem}
		There is a quillen adjunction 
		$$
		\leftone: \sset-\text{Cat} \rightleftarrows \tensorcrossedcat: \rightone
		$$
		where each functor is the identity on the object set and 
		for $\D \in \tensorcrossedcat$ the $\sset$-category $\rightone(\D)$ has $\rightone(\D)[x,y] = U_0(\D[x,y])$. 
	\end{theorem}
	\begin{proof}
		By \TODO{reference monidal omega}, $\crcom$ is a monoidal model category with respect to $\otimes$, and furthermore satisfies the monoid axiom. By \TODO{reference tonks}, the adjunction $\st_0 \dashv U_0$ is a weak monoidal quillen pair in the sense of \TODO{Schwede Shipley reference}. Thus this follows by \TODO{reference muro}.
	\end{proof} 
	Note that while $\rightone$ is given by applying $U_0$ locally, $\leftone$ is not. 
	\begin{example}
		Consider the $\sset$-category $\C$ given as the pushout in $\ssetcat$
		\begin{center}
		\begin{tikzcd}
		\bullet \ar[d, "\iota_1"] \ar[r, "\iota_0"] & \sus(\Delta^1) \ar[d, dashed]\\
		\sus(\Delta^1) \ar[r, dashed] & \C
		\end{tikzcd}
		\end{center}
		Thus $\C$ has three objects, which we call $\{0,1,2\}$, and $\C[0,1] \cong \C[1,2] \cong \Delta^1$, while $\C[0,2] \cong \Delta^1 \times \Delta^1$, while the other hom-objects are either empty or a point. 
		It is not hard to prove that for any simplicial set $X$, $\leftone(\sus(X)) \cong \sus(\st_0(X))$. So, since $\leftone$ must preserve pushouts, we have a pushout diagram
		\begin{center}
		\begin{tikzcd}
		\bullet \ar[d, "\iota_1"] \ar[r, "\iota_0"] & \sus(\st_0(\Delta^1)) \ar[d, dashed]\\
		\sus(\leftone(\st_0^1)) \ar[r, dashed] & \leftone(\C)
		\end{tikzcd}
		\end{center}
		And so $\leftone(\C)[0,2]$ must be $\Delta^1 \otimes \Delta^1$, which we described in example \ref{example:tensor-v-product-intervals}, in particular it has one generator in dimension $2$.
		On the other one, $\st_1(\C[0,2])$ is the crossed complex which has a fundamental groupoid formed by two noncommuting triangles
		\begin{center}
	\begin{tikzcd}[sep = large]
		0\otimes 0  \ar[r] \ar[dr] \ar[d] &  1 \otimes 0 \ar[d] \\
		0 \otimes 1 \ar[r] & 1 \otimes 1
	\end{tikzcd}
	\end{center}
	and two generators in dimension $2$ with boundaries the bottom left triangle and the upper right triangle, corresponding to the two nondegenerate simplices. 
	So, $\leftone(\C)[0,2] \not \cong \st_1(\C[0,2])$.
	However, they are weakly equivalent, as they are both contractible.
	\end{example}
	This example is just a special case of the fact that $\st_0 \dashv U_0$ is not strong monoidal. 
	If $\st_0 \dashv U_0$ were strong monoidal; in other words if we had that $\st_0(X \times Y) \cong \st_0(X) \otimes \st_0(Y)$, then we could compute $\leftone$ by applying $\st_0$ locally. 
	However, we do know that $\st_0 \dashv U_0$ is ``strong monoidal up-to-homotopy;'' i.e. in general we have a natural weak equivalence $\st_0(X \times Y) \to \st_0(X) \otimes \st_0(Y)$. 
	Hence, by \TODO{reference muro} we have the following:
	\begin{lemma}\label{lem:local-strictification-is-locally-strictification}
		Let $\C$ be a cofibrant $\sset$-category. Then for objects $x,y$ in $\C$, we have natural isomorphism in $\Ho(\crcom)$
		$$\st_1\left( \C[x,y] \right) \cong \leftone(\C)[x,y]$$
	\end{lemma}
	Hence, we can compute $\leftone$ up to local weak equivalence (just not up to isomorphism) by applying $\st_1$ to the hom-spaces. 
	We also note the following useful lemma:
	\begin{lemma}\label{lem:local-strictification-preserves-homotopy-category}
		Let $f: \C \to \D$ be a morphism in $\ssetcat$. Then we have a commutative square
		\begin{center}
		\begin{tikzcd}
			\Ho(\C) \ar[r, "\Ho(f)"] \ar[d, "\sim" {rotate=90, anchor=north}]
				& \Ho(\D) \ar[d, "\sim" {rotate=90, anchor=north}]
			\\
			\Ho(\leftone(\C)) \ar[r, "\Ho(\leftone(f))"] 
				& \Ho(\leftone(\D))
		\end{tikzcd}
		\end{center}
		in which the vertical arrows are isomorphisms of categories.
	\end{lemma}
	\begin{proof}
		This follows from the fact that $\leftone$ preserves the object set, and $\pi_0$ of the hom objects - the first fact being by definition, and the second fact because $\st_1$ preserves $\pi_0$ so by lemma \ref{lem:local-strictification} $\leftone$ preserves $\pi_0$ on the hom-objects, and hence the homotopy category.
	\end{proof}
\subsection{\TODO{} The adjunction between $\tensorcrossedcat$ and $\cartcrossedcat$}
	The adjunction
	$$
		\lefttwo: \tensorcrossedcat \rightleftarrows \cartcrossedcat : \righttwo
	$$
	is similar, but much less well-behaved than the adjunction $\leftone \dashv rightone$. 
	\begin{theorem}
		There is an adjunction 
		$$
		\lefttwo: \tensorcrossedcat \rightleftarrows \cartcrossedcat : \righttwo
		$$
		Where each functor is the identity on the object set, and for $\D \in \cartcrossedcat$ we have that $\righttwo(\D)[x,y] = \D[x,y]$. 
	\end{theorem}
	\begin{proof}
		The natural map $C \otimes D \to C \times D$ makes the identity functor on $\crcom$ a lax monoidal functor from $(\crcom, \times) \to (\crcom, \otimes)$. 
		Thus, $\righttwo$ as defined above exists, and the left adjoint exists by \TODO{reference muro and whatever muro references}. 
	\end{proof}
	Unlike in the previous case, we do not have a weak equivalence $C \otimes D \simeq C \times D$, and therefore we cannot approximate $\lefttwo$ via applying the identity hom-object-wise. 
	We also do not have the necessary hypotheses to immediately conclude that this is a quillen adjunction. However, this turns out to still follow via analysis.
	\begin{theorem}
		The adjunction of theorem \TODO{reference theorem} is a quillen adjunction
	\end{theorem}
	\begin{proof}
		% https://arxiv.org/html/2406.02194v1
		Using the link above, this boils down to showing that every object of $\tensorcrossedcat$ is fibrant. 
		As remarked by Muro, this does not follow immediately from the fact that every object of $\crcom$ is fibrant. 
		It must be proved somewhat manually. Hopefully this is not crazy hard, damn.
		\\\\
		I think this is not too bad: we construct a single generating interval which is given by just freely adjoining (in $\tensorcrossedcat$) an object, two morphisms, and two $1$-cells. 
		Then we need to prove that this is actually an interval, which boils down maybe to showing the cofibrant replacements are correct? Probably relevant is the stuff in muro's paper about showing cofibrancy sort of carries downward. 
		Then we need to show this is generating, which should be the obvious part because a map out of it is exactly the data to show that $x \cong y$.
	\end{proof} 
	
	\begin{lemma}\label{lem:global-strictification-preserves-homotopy-category}
		Let $f: \C \to \D$ be a morphism in $\tensorcrossedcat$. Then we have a commutative square
		\begin{center}
		\begin{tikzcd}
			\Ho(\C) \ar[r, "\Ho(f)"] \ar[d, "\sim" {rotate=90, anchor=north}]
				& \Ho(\D) \ar[d, "\sim" {rotate=90, anchor=north}]
			\\
			\Ho(\lefttwo(\C)) \ar[r, "\Ho(\lefttwo(f))"] 
				& \Ho(\lefttwo(\D))
		\end{tikzcd}
		\end{center}
		in which the vertical arrows are isomorphisms of categories.
	\end{lemma}
	\begin{proof}
		This follows from the fact that $\lefttwo$ preserves the object set, and $\pi_0$ of the hom-objects - this second fact is not obvious, but follows from the definition of the hom-objects of $\lefttwo(\C)$ as quotients.
	\end{proof}

\subsection{The adjunctions with $\grpdcat$}
	For completeness describe briefly the three vertical adjunctions of Diagram \ref{diagram:enriched-functors}. The three functors called $\Pi_1$ are all strong monoidal, and hence the functors $\hotwo$ are all given by applying $\Pi_1$ locally.
	\begin{theorem}
		The functors $\hotwo$ are each left Quillen. 
	\end{theorem}
	\begin{proof}
		This is a direct application of \TODO{reference muro}. 
	\end{proof}
\section{Conservativity: Two Special Cases}
	Our primary interest with strict higher categories is to study how much information is preserved or lost by the process of strictification. In analogy with the strictification of $\infty$-groupoids (spaces) as a nonabelian enrichment of homology, we would hope to prove theorems about the strictification of $(\infty,1)$-categories analagous to theorems about homology. The following theorem is a straightforward consequence of the classical theorem for spaces:
	\begin{theorem}
		$\leftone$ is conservative.
	\end{theorem}
	\begin{proof}
		This follows from Lemma \ref{lem:local-strictification-preserves-homotopy-category} and Theorem \ref{thm:crossed-complex-reflects-weak-equivalences}, applying Lemma \ref{lem:local-strictification-is-locally-strictification}.
	\end{proof}
	In the rest of this section we will prove that the functor $\lefttwo$ and hence also the composition $\st_1 = \lefttwo \circ \leftone$ reflect weak equivalences between two special types of objects: the ``$2$-truncated'' and the ``$2$-connected.'' We will also show as a consequence that $\st_1$ faithfully detects contractibility.
\subsection{2-truncated enriched categories}	
	First, let us recall the natural notions of truncatedness in $\sset$ and $\crcom$:
	\begin{definition}
		A simplicial set $X$ is called \emph{$n$-truncated} if for any $x \in X_0$ and $k >n$, the homotopy group $\pi_k(X,x)$ is trivial.
	\end{definition}
	\begin{definition}
		A crossed complex $C$ is called \emph{$n$-truncated} if for any $x \in C_0$ and $k >n$, the homotopy group $\pi_k(X,x)$ is trivial. 
	\end{definition}
	This extends naturally to enriched categories:
	\begin{definition}\label{dfn:n-truncated-ssetcat}
		Let $\C \in \ssetcat$. Then we call $\C$ \emph{n-truncated} if for any $x,y \in \ob \C$, the simplicial set $\C[x,y]$ is $(n-1)$-truncated.
	\end{definition}
	\begin{definition}\label{dfn:n-truncated-crossedcat}
		Let $\C \in \tensorcrossedcat$ or $\cartcrossedcat$. Then we call $\C$ \emph{n-truncated} if for any $x,y \in \ob \C$, the crossed complex $\C[x,y]$ is $(n-1)$-truncated.
	\end{definition}
	\begin{proposition}\label{prop:2-truncated-equivalent-sset}
		For $\C \in \ssetcat$, the following are equivalent:
		\begin{itemize}
			\item $\C$ is $2$-truncated in the sense of definition \ref{dfn:n-truncated-ssetcat}.
			\item The unit map $\C \to N \circ \hotwo(\C)$ is a weak equivalence.
		\end{itemize}
	\end{proposition}
	\begin{proof}
		($\Rightarrow$) The unit map $\C \to N \circ \hotwo(\C)$ is the identity on objects, and for any $x,y \in \ob\C$ the map $\C[x,y] \to N \circ \hotwo(\C)[x,y]$ is naturally isomorphic to the map $\C[x,y] \to N \circ \Pi_1(\C[x,y])$. If $\C$ is $2$-truncated, then this map is a weak equivalence. Since $N$ and $\C$ preserve the homotopy category, it follows that $\C \to N \circ \hotwo(\C)$ is a weak equivalence. 
		\par ($\Leftarrow$) If $\C \to N \circ \hotwo(\C)$ is a weak equivalence, then in particular for $x,y \in \ob\C$, $\C[x,y]$ is weak equivalent to $N \circ \Pi_1(\C[x,y])$ by the above. Since the latter is always $1$-truncated, the result follows.
	\end{proof}
	In precisely the same manner of proof we obtain:
	\begin{proposition}\label{prop:2-truncated-equivalent-tensorcrossed}
		For $\C \in \tensorcrossedcat$, the following are equivalent:
		\begin{itemize}
			\item $\C$ is $2$-truncated in the sense of definition \ref{dfn:n-truncated-crossedcat}.
			\item The unit map $\C \to \iota \circ \hotwo(\C)$ is a weak equivalence
		\end{itemize}
	\end{proposition}
	
	\begin{proposition}\label{prop:2-truncated-equivalent-cartcrossed}
		For $\C \in \cartcrossedcat$, the following are equivalent:
		\begin{itemize}
			\item $\C$ is $2$-truncated in the sense of definition \ref{dfn:n-truncated-crossedcat}.
			\item The unit map $\C \to \iota \circ \hotwo(\C)$ is a weak equivalence
		\end{itemize}
	\end{proposition}
	
	\begin{lemma}\label{lem:2-truncated-preserved}
		The functor $\leftone$ sends cofibrant $2$-truncated objects to $2$-truncated objects.
	\end{lemma}
	\begin{proof}
		This follows from the fact that $\st_0$ sends $1$-truncated objects to $1$-truncated objects, along with Lemma \ref{lem:local-strictification-is-locally-strictification}.
	\end{proof}
	Note that the above theorem is false for $n > 2$: the functor $\st_0$ does not preserve $n$-truncated objects generally, and hence neither can $\leftone$. 
	For a counterexample, consider any simplicial set $X$ that is a $K(\mathbb{Z},n)$ for $n > 1$; since the homotopy groups of $\st_0(X)$ are the homology groups of $X$, $\st_0(X)$ is not $n$-truncated. 
	Hence the simplicial category $\sus[X]$ will be $n+1$ truncated but $\leftone \sus[X]$ will not be. 
	A counterexample to $\lefttwo$ preserving $n$-truncated objects for $n > 2$ is trickier to construct. 
	We will give an example later when we analyze the behaviour of $\lefttwo$ on $2$-connected objects. We believe that $\lefttwo$ does not preserve $2$-truncated objects, but lack a good enough description to provide an easy example. \TODO{it's possible I can; I should try for a little bit} 
	\par 
	The following special case is not difficult but is noteworthy; it reflects a general trend that weak categories in dimension $\le 2$ are captured up to equivalence by strict categories. 
	\begin{theorem}
		If $\C \in \ssetcat$ is $2$-truncated and cofibrant, then the unit $\C \to \rightone \leftone (\C)$ is a weak equivalence 
	\end{theorem}
	\begin{proof}
		By commutativity of Diagram \ref{diagram:enriched-functors}, we have a commutative up-to-isomorphism triangle
		\begin{center}
		\begin{tikzcd}
			\C \ar[rr] \ar[rd] 
				& & \rightone \circ  \iota \circ  \hotwo \circ \leftone (\C) \ar[ld] 
			\\
			& N \circ \hotwo (\C)
		\end{tikzcd}
		\end{center}
		And then by Propositions \ref{prop:2-truncated-equivalent-sset}, \ref{prop:2-truncated-equivalent-tensorcrossed} 2 of the arrows are weak equivalences, so the third is as well (we have used that $\rightone$ preserves weak equivalences, which follows as it is given by locally applying $U_0$, which preserves weak equivalences). 
	\end{proof}
	The following theorem is the precise incarnation of the first special case of conservativity; saying that $\lefttwo \circ \leftone$ is conservative on the $(2,1)$-categories.
	\begin{theorem}
		The functors $\leftone$, $\lefttwo$, and $\lefttwo \circ \leftone$ reflect weak equivalences between cofibrant, $2$-truncated objects.
	\end{theorem}
	\begin{proof}
		The first two statements follow from applying Propositions \ref{prop:2-truncated-equivalent-sset}, \ref{prop:2-truncated-equivalent-tensorcrossed}, \ref{prop:2-truncated-equivalent-cartcrossed}. The statement about $\lefttwo \circ \leftone$ follows from the other two via Lemma \ref{lem:2-truncated-preserved}. 
	\end{proof}
\subsection{2-connected quasicategories, and detecting contractability}
	We will now prove a dual special case. First, we recall the dual notion of truncatedness, connectivity: 
	First, let us recall the natural notions of truncatedness in $\sset$ and $\crcom$:
	\begin{definition}
		A simplicial set $X$ is called \emph{$n$-connected} if for any $x \in X_0$ and $0\le k \le n$, the homotopy group $\pi_k(X,x)$ is trivial.
	\end{definition}
	\begin{definition}
		A crossed complex $C$ is called \emph{$n$-connected} if for any $x \in C_0$ and $0 \le k \le n$, the homotopy group $\pi_k(X,x)$ is trivial. 
	\end{definition}
	This extends naturally to enriched categories:
	\begin{definition}\label{dfn:n-connected-ssetcat}
		Let $\C \in \ssetcat$. Then we call $\C$ \emph{n-connected} if for any $x,y \in \ob \C$, the simplicial set $\C[x,y]$ is $(n-1)$-connected.
	\end{definition}
	\begin{definition}\label{dfn:n-truncated-crossedcat}
		Let $\C \in \tensorcrossedcat$ or $\cartcrossedcat$. Then we call $\C$ \emph{n-connected} if for any $x,y \in \ob \C$, the crossed complex $\C[x,y]$ is $(n-1)$-connected.
	\end{definition}
	In particular, an enriched category is ``$0$-connected'' when every hom-space is nonempty, and ``$1$-connected'' when the homotopy category is a contractible groupoid. 
	\begin{lemma}\label{lem:local-strictification-preserves-connectedness}
		$\leftone$ sends $n$-connected objects to $n$-connected objects.
	\end{lemma}
	\begin{proof}
		This follows from the definition of $\leftone$ and the fact that $\st_0$ sends $(n-1)$-connected objects to $(n-1)$-connected objects. 
	\end{proof}
	\TODO{decide whether the rest is worth stating}. 
	We will not use the following lemma explicitly byt the result is worth noting. 
	\begin{lemma}\label{lem:global-strictification-preserves-connectedness}
		$\lefttwo$ sends $n$-connected cofibrant objects to $n$-connected objects.
	\end{lemma}
	\begin{proof}
		The cases $n \le 2$ follow from the commutativity of \ref{diagram:enriched-functors}. So suppose $n > 2$, and $\C \in \tensorcrossedcat$ is $n$-connected and cofibrant. In particular, its homotopy category is a contractible groupoid, so we can take any object $x \in \C$ and consider the category $\C_x$ with one object $x$ and $\C_x[x,x] = \C[x,x]$. This has a natural inclusion $\C_x \hookrightarrow \C$, which is a weak equivalence. Now since $\C_x$ is $n$-connected, we can approximate it via a cofibrant $n$-reduced object $E_n(\C_x)$, where by $n$-reduced we mean that it has a single object $x$ and the endomorphism crossed complex $E_n(\C_x)[x,x]$ is $(n-1)$-reduced. It follows that $\lefttwo E_n(\C_x)[x,x] \to \lefttwo\C$ is a weak equivalence, since the former is clearly $n$ reduced the result follows
	\end{proof}
	We can now prove the other special case of conservativity, for the $2$-connected objects:
	\begin{theorem}
		The functors $\lefttwo$, and $\lefttwo \circ \leftone$ reflect weak equivalences between cofibrant, $2$-connected objects.
	\end{theorem}
	\begin{proof}
		As usual, since $\leftone$ reflects weak equivalences between all cofibrant objects, it suffices to prove for $\lefttwo$. By approximation with $2$-reduced objects as in the proof of Lemma \ref{lem:global-strictification-preserves-connectedness} and by applying the small object argument we can reduce to the case of an $I_{>2}$-cellular map $\C \to \D$, where $I_{>2}$ are the generating cofibrations for $\tensorcrossedcat$ which are $2$-reduced. 
		\par Now, for such a $\C$ and $\D$, we equip the unique hom-object $\C[x,x]$ with the cofiltration $T_*(\C[x,x])$, where $$
	\end{proof}
\section{Universality}
	In this short section we describe informally the model-independence of the functor $\st_1$, utilizing the universal description of the homotopy theory of $(\infty,n)$-categories given in \TODO{reference.}
	First, we recall the universal property of the homotopy theory of spaces, due in the model category sense to \TODO{dugger universal homotopy theories} and in the quasicategorical sense to \TODO{HTT 5.1.5.6}.
	\begin{theorem}
		Let $\S$ be the quasicategory of (unpointed) spaces, and $\C$ any quasicategory. A left adjoint $F: \S \to \T$ is determined up to contractible choice by the object $F(\bullet)$, where $\bullet$ is the single-point space.
	\end{theorem}
	This is generally summarized by saying that the homotopy theory of spaces is the ``free cocomplete homotopy theory on a point.'' 
	This allows a distinct lack of ambiguity when talking about left adjoint functors out of $\S$; 
	merely selecting an object in the target category is sufficient to uniquely specify one such functor. 
	Since our constructions in this paper have been very model specific, we would like to argue that in some sense our functors $\st_1$ is universal. 
	In \TODO{reference}, the authors give a universal characterization of the homotopy theory of $(\infty,n)$ categories, generalizing that of $\S$ (which is the special case $n=0$). We recall just the portion of their argument we need here:
	\begin{theorem}
		Let $\S_n$ the quasicategory of $(\infty,n)$-categories, and $\iota: \text{Gaunt}_n \hookrightarrow \S_n$ the inclusion of the gaunt $n$-categories. For any other cocomplete quasicategory $\C$ and functor $k: \text{Gaunt}_n \to \C$, there is up to equivalence at most one functor $\S_n \to \C$ which makes the diagram
		\begin{center}
		\begin{tikzcd}
			\S_n \ar[r, dashed] & \C \\
			\text{Gaunt}_n \ar[u, "\iota "] \ar[ru, "k"]
		\end{tikzcd}
		\end{center}
	commute.
	\end{theorem}
	The proof is a special case of Lemma 9.2 in \TODO{unicity}. 
	\begin{theorem}
		The derived functor of $\st_1$ preserves gaunt $1$-categories (up to natural weak equivalence). 
	\end{theorem}
	\begin{proof}
		Let $\C \in \ssetcat$ be a gaunt $1$-category. Take the standard cofibrant resolution $\C^{cof} \to \C$ by the simplicial computad $\C^{cof}$ which has a generating $n$-simplex in $\C[x,y]$ for every $(n+1)$-tuple of morphisms $x \to \cdots \to y$. Since $\st_1$ preserves pushouts, 
		\TODO{see polygraphs from rewriting to higher categories 23.3.8}.
	\end{proof}
	From this we get:
	\begin{corollary}
		Let $(\infty,1)\mathcal{Cat}$ and $(omega,1)\mathcal{Cat}$ be the quasicategories induced by localizing the corresponding model structures. Then the induced functor $\st_1 :(\infty,1)\mathcal{Cat} \to (omega,1)\mathcal{Cat}$ can be uniquely defined as the functor which sends each gaunt $n$-category to itself.
	\end{corollary}
	
\section{The word-length cofiltration and its associated spectral sequence}
	Throughout this section, fix $\C \in \tensorcrossedcat$ and fix $x,y$ be objects of $\C$.
	\begin{definition}	
		For $n \ge 2$ and $a \in \C[x,y]_n$, we say that $a$ has \emph{homogeneous word length $\ge k$} if there are objects $x = x_0,x_1,...,x_k = y$, integers $\ell_i \ge 1$, and elements $a_i \in \C[x_{i-1},x_i]_{\ell_i}$  such that the composition
		$$\circ : \C[x_0,x_1] \otimes \cdots \otimes \C[x_{k-1}, x_k] \to \C[x_0,x_k] = \C[x,y]$$
		carries $a_1 \otimes \cdots \otimes a_k$ to $a$.
		We say that $a$ has \emph{word length $\ge k$} if it can be written as a sum of elements of homogeneous word length $\ge k$.
		\end{definition}
	 For $x,y \in \ob(\C)$, we can cofilter the crossed complex $\C[x,y]$ according to the word length: for each $n$, let $F_k\C[x,y]$ be the crossed complex which is the quotient of $\C[x,y]$ by all elements of word length $> k$. This gives us the cofiltration
	 \begin{center}
	 \begin{tikzcd}[sep = small]
	 	\C[x,y] \ar[r, two heads] & \cdots \ar[r, two heads] & \C[x,y]_{2} \ar[r, two heads] & \C[x,y]_1 \cong *
	 \end{tikzcd}
	 \end{center}
	 \begin{lemma}
	 	The quotient map $\C_{k} \to \C_{k-1}$ is $k-1$ connected.
	 \end{lemma}
	 \begin{proof}
	 	By definition, this map is given by quotienting out elements of degree at least $k$, from which this follows.
	 \end{proof}
	 From this cofiltration we get an associated spectral sequence. Our aim will be to show that the $E^2$ page of this spectral sequence can be computed from $\st \C$, as a homotopy invariant (with cofibrancy assumptions). 
	 From this the main result will follow. 
	 \begin{lemma}
	 	$\C[x,y]_1 \cong \st(\C)$.
	 \end{lemma}
	 \begin{proof}
	 	This follows from the description of the adjoint functor given in the appendix.
	 \end{proof}
	In order to analyze the pages, we need to understand the kernel of the maps $\C[x,y]_k \twoheadrightarrow \C[x,y]_{k-1}$. 
	We will now make the simplifying assumption that $\C$ is a computad.
	The kernel therefore is the chain complex of tensors $a_1 \otimes \cdots \otimes a_n $ such that exactly $k$ of the $a_i$ have dimension strictly positive. Call this group $F_k(\C)$
	\begin{lemma}
		Let $\C \in \tensorcrossedcat_\mathcal{I}$ be a computad and $\C \to \D$ a relative computad. 
		If $\st(\C) \to \st(\D)$ is a weak equivalence, then the maps $F_k(\C) \to F_k(D)$ are homology equivalences.
	\end{lemma}
	\begin{proof}
		We will describe these homology groups as the homology groups of the crossed complex which is given by the hom-object of the $k$-fold tensor product, pushed forward over the groupoid composition map. In fact, maybe first we should just quotient everything out into being over groups, to make our lives easier. And then we'll have a spectral sequence over every morphism in the homotopy category, of G modules (G being the fundamental group of that morphism automorphism space). And the subfibers are the k-fold tensors which have base point equivalent to that element of the homotopy category, and these come from writing (in the homotopy category) $f$ as a $k$-fold composition, and over each one selecting a simple tensor (degree 1). Ooh!
	\end{proof}
	
\subsection{\TODO{clean up} The fundamental crossed complex of a simplicial set}
	We have referenced already the strong connection between homology and $\omega$-groupoids. In fact, the relation is extremely strong: there is a functor $R: \ch \to \omega Grpd$ which takes the chain complex $C_\bullet$ to an $\omega$-groupoid with underlying globuler set
	\TODO{make this nicer}
	\begin{center}
	\begin{tikzcd}
	C_0 & C_0 \times C_1 & C_0 \times C_1 \times C_2 & \cdots
	\end{tikzcd}
	\end{center}
	And with the composition operation on
	\begin{center}
	\begin{tikzcd}
	C_0 \times \cdots \times C_i & C_0 \times  \cdots \times C_i \times \cdots \times C_{i + k}
	\end{tikzcd}
	\end{center}
	given by addition on the last $k$ factors. This functor $R$ should be thought of as interpreting each dimension $C_i$ as a collection of ``unbased $i$ dimensional morphisms,'' which paired with a source (an $i-1$ morphism; inductively an element of $C_0 \times \cdots \times C_{i-1}$) gives rise to an $i$ morphism.\footnote{Those familiar with the Dold-Kan construction should compare this, which has an extremely similar definition - which is no accident; this definition unwinds a chain complex into a globular set; Dold-Kan unwinds a chain complex into a simplicial one instead.} Surprisingly, it turns out that there is a partial inverse to this operation: given an arbitrary $\omega$ groupoid $G$, there is a process to ``distill'' out the ``unbased'' data in each dimension, leaving one with a sort of nonabelian chain complex called a \textit{crossed complex}.
	\TODO{The crossed complex of an $\omega$ groupoid} \\
	It turns out that this functor is actually an equivalence of categories: from the unbased morphism data of $afhadfkj$, the original $\omega$ groupoid can be recovered, not just up to weak equivalence but up to isomorphism. While we are conceptually interested in $\omega$ groupoids as the strict analogue of $\infty$ groupoids, it is clear that crossed complexes provide a convenient framework in which to do computation and proof. We therefore from now on shall be entirely concerned with constructing an adjunction from adklfankldsfk to asdlfmnalsdkf, while aware that our results imply identical results for $(\omega,1)$ categories as described above. 
	\TODO{brief clarification on this. Make sure to be concrete about how the functors are defined and the various models - I think in this paper I will only really need the one sSet -> omega-groupoid. Recall the theorems from Tonks for ease of reference.} 
	
\subsection{Proof of theorem \TODO{theorem numbering}}
	Having set up our principle diagram, we can now state and prove our main theorem:
	\begin{theorem}
		Both left adjoints in the diagram 
		\begin{center}
		\begin{tikzcd}[sep = large]
			\ssetcat \ar[r, bend left = 20, "\leftone "] 
				& \tensorcrossedcat \ar[l, bend left = 20, "\rightone "] \ar[r, bend left = 20, "\lefttwo "] 
				& \cartcrossedcat \ar[l, bend left = 20, "\righttwo "]
		\end{tikzcd}
		\end{center}
		reflect weak equivalences between cofibrant objects. In particular, their composition reflects weak equivalences between cofibrant objects.
	\end{theorem}
	\begin{proof}
		We first analyze $\leftone$: let $f: \C \to \D$ be a morphism in $\ssetcat$, with $\C$ and $\D$ cofibrant. 
		If $\leftone(f)$ is a weak equivalence, then it is an equivalence on homotopy categories, and therefore so is $f$ by lemma \TODO{reference lemma}. 
		Hence it suffices to prove that $f$ induces weak equivalences on hom spaces. By lemma \TODO{reference lemma}, for $x,y \in \ob(\C)$ we have a commutative square 
		\begin{center}
		\begin{tikzcd}
			\st_0(\C[x,y]) \ar[r] \ar[d] 
				& \st_0(\D[f(x), f(y)]) \ar[d]
			\\
			\leftone(\C)[x,y] \ar[r]
				& \leftone(\D)[f(x), f(y)]
		\end{tikzcd}
		\end{center}
		in which the verticle arrows are weak equivalences. 
		By assumption, the bottom arrow is a weak equivalence, so the top one is as well. 
		The result then follows by theorem \TODO{reference strictification 0 conservative theorem}.
		\\ \par 
		We now move on to proving that $\lefttwo$ is conservative: let $f: \C \to \D$ be a morphism between cofibrant objects in $\tensorcrossedcat$. 
		Suppose that $\lefttwo(f)$ is a weak equivalence in $\cartcrossedcat$. 
		Then by lemma \TODO{reference lemma} $f$ is an equivalence on homotopy categories.
		Define a new object of $\tensorcrossedcat$, $\C^{\text{Mor}(f)}$, which has the same objects as $\C$ and such that 
		$$\C^{\text{Mor}(f)}[x,y] := \D[fx,fy]$$
		With composition maps given by the composition maps in $\D$. Then $f$ factors as 
		\begin{center}
		\begin{tikzcd}
			\C \ar[r] & \C^{\text{Mor}(f)} \ar[r, "{\sim}"] & \D
		\end{tikzcd}
		\end{center}
		Where the second map is a weak equivalence, being homotopically essentially surjective (because $f$ is) and evidently a local weak equivalence. 
		Now, let $Q$ be a cofibrant replacement functor which is identity-on-objects (\TODO{I MIGHT NEED TO ACTUALLY DO SOMETHING HERE IDK} for instance, $Q$ can be the composition of the homotopy coherent nerve and rigidication functors \TODO{reference}). We obtain a diagram
		\begin{center}
		\begin{tikzcd}
			Q(\C) \ar[r] \ar[d, "\sim " {rotate=90, anchor=north}] & Q\left(\C^{\text{Mor}(f)}\right) \ar[r, "{\sim}"] \ar[d, "\sim " {rotate=90, anchor=north}] & Q(\D) \ar[d, "\sim "] \\
			\C \ar[r] & \C^{\text{Mor}(f)} \ar[r, "{\sim}"] & \D
		\end{tikzcd}
		\end{center}
		Now, we can further factor the top left map as a cofibration followed by an acylic fibration. This gives us the diagram
		\begin{center}
		\begin{tikzcd}
			Q(\C) \ar[r] \ar[d, "\sim " {rotate=90, anchor=north}] & A \ar[r, "\sim "] & Q\left(\C^{\text{Mor}(f)}\right) \ar[r, "{\sim}"] \ar[d, "\sim " {rotate=90, anchor=north}] & Q(\D) \ar[d, "\sim " {rotate=90, anchor=north}] \\
			\C \ar[rr]& & \C^{\text{Mor}(f)} \ar[r, "{\sim}"] & \D
		\end{tikzcd}
		\end{center}
		Now, applying $\lefttwo$ to the above diagram, we get a diagram
		\begin{center}
		\begin{tikzcd}
			\lefttwo Q(\C) \ar[r] \ar[d, "\sim " {rotate=90, anchor=north}] & \lefttwo  A \ar[r, "\sim "] & \lefttwo Q\left(\C^{\text{Mor}(f)}\right) \ar[r, "{\sim}"] \ar[d] & \lefttwo Q(\D) \ar[d, "\sim " {rotate=90, anchor=north}] \\
			\lefttwo  \C \ar[rr]& & \lefttwo  \C^{\text{Mor}(f)} \ar[r] & \lefttwo\D
		\end{tikzcd}
		\end{center}
		Where the composition of the bottom two arrows is a weak equivalence. It then follows that top left arrow is a weak equivalence. \
		Note that if the map $Q(\C) \to A$ is a weak equivalence, then so is $\C \to \D$. 
		We have therefore reduced the problem to showing that $\lefttwo$ reflects weak equivalences on identity-on-objects relative $I$-cellular maps between $I$-cellular objects of $\tensorcrossedcat$, with $I$ being the generating cofibrations.
		So, let $f: \C \to \D$ be such a map, and suppose that $\lefttwo(f)$ is a weak equivalence. 
		Then by \TODO{reference spectral sequence} for any objects $x,y \in \ob(\C)$ we have spectral sequences converging to ------, and a natural transformation $\phi$ of spectral sequences. 
		By \TODO{reference}, since $\lefttwo(f)$ is a weak equivalence, this natural tranformation is an isomorphism of $E^2$-pages, and hence of $E^\infty$-pages. 
		Thus $f$ induces isomorphisms on $\pi_n$ for $n \ge 2$. 
		Since we have already shown it induces an equivalence of homotopy categories and of local fundamental groupoids, the result then follows.
	\end{proof}
\section{OLD: Conservativity}
	\begin{theorem}
		Let $\C$ and $\D$ be simplicial computads and $f: \C \to \D$ a functor of simplicially enriched categories. Then if $L(f)$ is a weak equivalence, so is $f$. 
	\end{theorem}
	\begin{proof}
		Firstly, note that if $L(f)$ is a weak equivalence, then it is an equivalence on homotopy categories, and therefore so is $f$ (since $\text{St}_1$ preserves homotopy categories). 
		Define a simplicially enriched category $\C^{\text{Mor}(f)}$ which has the same objects as $\C$ and such that 
		$$\C^{\text{Mor}(f)}[x,y] := \D[fx,fy]$$
		With composition maps given by the composition maps in $\D$. Then $f$ factors as 
		\begin{center}
		\begin{tikzcd}
			\C \ar[r] & \C^{\text{Mor}(f)} \ar[r, "{\sim}"] & \D
		\end{tikzcd}
		\end{center}
		Where the second map is a weak equivalence. Now, let $Q$ be a cofibrant replacement functor which is identity-on-objects (for instance, $Q$ can be the composition of the homotopy coherent nerve and rigidication functors \TODO{reference}). We obtain a diagram
		\begin{center}
		\begin{tikzcd}
			Q(\C) \ar[r] \ar[d, "\sim "] & Q\left(\C^{\text{Mor}(f)}\right) \ar[r, "{\sim}"] \ar[d, "\sim "] & Q(\D) \ar[d, "\sim "] \\
			\C \ar[r] & \C^{\text{Mor}(f)} \ar[r, "{\sim}"] & \D
		\end{tikzcd}
		\end{center}
		Now, we can further factor the top left map as a cofibration followed by an acylic fibration. This gives us the diagram
		\begin{center}
		\begin{tikzcd}
			Q(\C) \ar[r] \ar[d, "\sim "] & A \ar[r, "\sim "] & Q\left(\C^{\text{Mor}(f)}\right) \ar[r, "{\sim}"] \ar[d, "\sim "] & Q(\D) \ar[d, "\sim "] \\
			\C \ar[rr]& & \C^{\text{Mor}(f)} \ar[r, "{\sim}"] & \D
		\end{tikzcd}
		\end{center}
		Now, applying $\st$ to the above diagram, we get a diagram
		\begin{center}
		\begin{tikzcd}
			\st Q(\C) \ar[r] \ar[d, "\sim "] & \st A \ar[r, "\sim "] & \st Q\left(\C^{\text{Mor}(f)}\right) \ar[r, "{\sim}"] \ar[d] & \st Q(\D) \ar[d, "\sim "] \\
			\st \C \ar[rr]& & \st \C^{\text{Mor}(f)} \ar[r] & \st\D
		\end{tikzcd}
		\end{center}
		Where the composition of the bottom two arrows is a weak equivalence. It then follows that top left arrow is an isomorphism. Note that if the map $Q(\C) \to A$ is a weak equivalence, then so is $\C \to \D$. 
		We have therefore reduced the problem to showing that $\st$ reflects weak equivalences on identity-on-objects cofibrations between cofibrant sset-categories, i.e. relative simplicial computads which adjoin no additional objects. 
		So, let $f: \C \to \D$ be such a map, and suppose that $\st(f)$ is a weak equivalence. Then by \TODO{clean up all of the rest of this} [] it is $J_1$-cellular, hence $\st(f)$ can be written as adjoining generators which are partitioned and ordered as in [].
		Of course, since $\st$ is the image of a relative computad and hence (since $\st$ is a left adjoint) itself a pushout of ta diagram of the same shape, $\st(f)$ can also be written as adjoining generators which are the images of the simplices adjoined to $\C$ to create $\D$. These two identifications of $\st(D)$ give us an isomorphism between the two relative $(\omega,1)$-computads which fixes the subcomputad $\st(\C)$. This allows us to also define an isomorphism between the respective $\tensorcrossedcat$-enriched categories, and therefore demonstrates that this weak strictification of $f$ is isomorphic to an acylic cofibration and is acylic. Since the weak strictification reflects weak equivalences, it follows that $f$ is a weak equivalence. 
	\end{proof}
\subsection{a thought on the above}
		Note that this suggests that the suspension of an $\infty$-groupoid into a quasicategory is strictified to simply be the suspension of the strictification of original $\infty$-groupoid - that is, strictification and suspension commute. But this should be expected as the loops appear to commute. Or maybe it should be unexpected and that means the right adjoint is incorrect? Or it's expected because suspension is a homotopy colimit...

\subsection{another thought}
I think that we automatically get Quillen adjunction if this is an adjunction at all, though this requires essentially showing that a fibration of $(\omega,1)$ categories is one which is a fibration on all hom-sets. (and fibration of homotopy categories). But I think this is not hard.

\subsection{note on above and the groupoid case}
	It is tempting to imagine that one could use this same technique to prove whitehead's homological theorem: perhaps replicating the acylic cofibration reflection argument it can be done. However, this cannot work, because of the existence of non-simple homotopy equivalences: if it were possible to prove that if $f: X \to Y$ is an cofibration of simplicial sets, then $\st(f)$ is a w.e. $\implies$ $f$ is $J$-cellular, then this would in particular imply that if $f$ is a w.e. then it is $J$-cellular, which is not true (because of the existence of non-simple homotopy equivalences).
	
\subsection{Can we get to... comonadicity??}
	\TODO{the basic argument is that once we have conservativity, we just need to show that the map into the canonical cosimplicial resolution is an equivalence. This maybe isn't too hard: need to show that it is an equivance on homotopy categories, and on every mapping space. The first part should follow because St doesn't change the homotopy category, and the second should follow using that mapping spaces are homotopy limits, and that St0 is comonadic}

\section{Looking upwards: difficulties in defining the strictification of $(\infty,n)$-categories, comparison with Loubaton-Henry}

\subsection{Proving that this adjunction is ``universal'' (takes the gaunt n-categories to themselves)}
	The thing to do here is to take the canonical resolution and try to claim that it gives a cofibrant resolution in $(\omega,1)$-cat, i.e. prove that a specific weak equivalence (whose codomain is \textit{not} cofibrant) is preserved. 
\subsection{Monoidalness of the strictification adjunction for quasicategories}
	An essential tool in our approach to defining $\text{St}_1$ is that the adjunction $\text{St}_0 \dashv U_0$ is (lax) monoidal, 
	\TODO{try to figure out whether the strictification of quasicategories is monoidal, which would give the $(\infty,2)$-categorical strictification immediately}
\subsection{Looking upwards}
	Lay out whatever the next logical step is and then explain why it is difficult/not possible with the current technology.	
\subsubsection{monoidal structures on higher categories}
	There is a monoidal product on $\omegancat{n}$ called the \textit{gray tensor product}.
	Originally defined for $2$-categories in [\TODO{reference}], it was extended to $\omega$ and $(\omega,n)$-categories in []. 

\subsection{Comparison with Loubaton-Henry}
		Henry says an adjustment to the marking might give you a model structure on marked $\omega$-categories modelling $(\omega,n)$-categories. Attempt to do this, or at least to define a functor from his model category to ours, and use this to prove his conjecture in the $n = 0,1$ cases.
		\TODO{Determine whether the functor that inverts everything about dimension $n$ will give a left Quillen functor from Loubaton/Henry to (omega,n)-categories, adjoint to some natural inclusion (marking not obvious). Other option: construct another model structure on omega-categories in which the fibrant objects are the (omega,n)-categories.}
		\TODO{demonstrate that this gives a factorization of our strictification functor through theirs. Conclude that their strictification functor is conservative.}
		
\section{Appendix: Explicit Descriptions of Left Adjoint $\ssetcat \to \tensorcrossedcat$ }
	\subsection{The left adjoint -----}
	The goal of this section is to expand upon the results in [Tonks], generalizing them all simplicial computads. Recall that the auther defines functions $a: \pi(X \times Y) \to \pi X \otimes \pi Y$ and $b: \pi X \otimes \pi Y \to \pi (X \times Y) $, and proves that $ab = \id$ and $ba$ is a deformation retraction. We view this as a special case of a simplicial computad in the following way: let $\mathds{1}[X,Y]$ be the simplicial computad given by the pushout
	\begin{center}
	\begin{tikzcd}
		\bullet \ar[r, "0"] \ar[d, "1"] 
			& \mathds{1}[Y] \ar[d] \\
		\mathds{1}[X] \ar[r] 
			& \mathds{1}[X,Y]
	\end{tikzcd}
	\end{center}
Then the data of $a$ and $b$, along with identity maps, assembles into functors in $\tensorcrossedcat$
$$A: \text{St}_1\mathds{1}[X,Y] \rightleftarrows \mathds{1}[\pi X, \pi Y]: B$$
With $AB = \id$ and $BA$ a deformation retract of the identity functor. This demonstrates that the pushout square above is preserved up-to-homotopy by $\text{St}_1$, which is what suggests that $\text{St}_1$ may present a left adjoint of higher $(\infty,1)$ categories. $\mathds{1}[X,Y]$ is a special case of a simplicial computad, and we will provide a generalization of the above weak-pushout-preservation to all simplicial computads. 
\\\\
\begin{theorem}
	Let $\C$ be a simplicial computad, given as the iterated pushout along maps described in [Riehl], one for each of its atomic morphisms, so $\C = \colim_{\Delta^n \in \text{at}\C }\mathds{1}[\Delta^n]$. Then define $\mathcal{F}\langle \text{at}\C\rangle \in \tensorcrossedcat$ to be the iterated pushout where each of those maps is replaced by its image under $\text{St}_1$. Then there are morphisms $A$ and $B$,
	$$A: \text{St}_1\C \rightleftarrows \mathcal{F}\langle\text{at} \C \rangle: B$$
	Such that $AB$ is the identity and $BA$ is a deformation retract of the identity. Moreover, these are natural in the sense that if $\D$ is another simplicial computad, and $F: \C \to \D$ a morphism of simplicial computads (sends atoms to atoms), then we have commutative squares 
	\begin{center}
	\begin{tikzcd}
		\C \ar[r, "F"]
			& \D \\
		\mathcal{F}\langle \emph{at}\C\rangle \ar[r, "\hat{F}"] \ar[u, "B"]
			& \mathcal{F}\langle \emph{at}\D \rangle \ar[u, "B"]
	\end{tikzcd}
	\end{center}
	and
	\begin{center}
	\begin{tikzcd}
		\C \ar[r, "F"] \ar[d, "A"]
			& \D \ar[d,"A"] \\
		\mathcal{F}\langle \emph{at}\C\rangle \ar[r, "\hat{F}"] 
			& \mathcal{F}\langle \emph{at}\D \rangle 
	\end{tikzcd}
	\end{center}
	Where $\hat{F}$ here means the morphism induced by the morphism of diagrams which is the image of the morphism of diagrams inducing $F$ under $\text{St}_1$. \TODO{terrible phrasing}.
\end{theorem}
The proof will span this appendix.
\\\\
First, a little notation: for $X$ a simplicial set and a simplex $x \in X_n$, and integers $0\le a_0 \le ... \le a_k \le n$, we will continue to use the notation $a_{a_0\cdots a_k}$ denote the simplex given by the composition
\begin{center}
\begin{tikzcd}
\Delta^k \ar[r, "(a_i)"]
	& \Delta^n \ar[r, "x"] 
	& X
\end{tikzcd}
\end{center}
Where $(a_i)$ is the representable map induced by the map $[k] \to [n]$ that sends $i$ to $a_i$. For our purposes, a \emph{overlapping partition into $k$ parts} of the ordered set $[n]$ is a list $D$ of ordered subsets $(\{0 \le \cdots \le a_1\}, \{a_1 \le \cdots \le a_2\}, \cdots \{a_{k-1} \le a_k\})$ such that each element of $[n]$ appears at least once (equivalently, in each ordered subset-with-possible repetition $\{a_{i-1}, \cdots, a_i\}$, no elements are ``skipped.'') We define $D_i = \{a_{i-1}, \cdots, a_i\}$. We use the notation $x_{D_i}$ to mean $x_{a_{i-1}\cdots a_i}$. The set of overlapping partitions into $k$ parts of the ordered set $[n]$ is denoted by $[n]_k$. For a fixed $n$, $i$ and $k$ with $i + k \le n$, we will also use $d[i \to i + k]$ to denote the map $[k] \to [n]$ given by $j \mapsto j + i$ - in other words, the inclusion starting at $i$ with no skips.
\\\\
For $p_1,...,p_n$ nonnegative integers, we define $S_{p_1,...,p_n}$ to be the set of tuples $\sigma = (\sigma_0,...,\sigma_n)$ of functions $\sigma_i : \{1 \le \cdots \le p_i\} \to \{1 \le \cdots \le p_1 + \cdots + p_n\}$ with disjoint images. 
Given such a $\sigma$, we define $s_{\sigma_i}$ to be the degeneracy operator induced by the map $f: [p_1 + \cdots + p_n] \to [p_i]$ defined as 
$$ f(0) = 0 \quad \text{and} \quad f(i) =  
\begin{cases}
f(i-1) & \text{if } i \not \in \text{im}(\sigma_i) \\
f(i-1) + 1 & \text{if } i \in \text{im}(\sigma_i)
\end{cases}$$
We furthermore define the \textit{sign} of $\sigma$, $\text{sgn}(\sigma)$, to be the sign of the permutation given by listing out the images of each $\sigma_i$, starting with $\sigma_1$. 
There is a bijection between $S_{p_1,...,p_n}$  and the set of nondegenerate top dimensional simplices of $\Delta[p_1] \times \cdots \times \Delta[p_n]$, given by $\sigma \mapsto (s_{\sigma_1} x_1,...,s_{\sigma_n} x_n)$, where $x_i$ is the unique nondegenerate $p_i$-simplex of $\Delta[p_i]$. 
Note that our notation in the case $n=2$ differs from [Tonks] slightly.
\\\\
\subsection{The functor $B$}
	The functor $B: \mathcal{F}\langle \text{at}\C\rangle \to \text{St}_1\C$ is simple to define: as its source is a pushout, it suffices to define the values on the generators. So, for any atom $\alpha \in \freeatom{\C}[p,q]_n$, we recall that $\alpha$ corresponds to an atom $\tilde{\alpha} \in \C$ and simply let $B(\alpha) = \tilde{\alpha}$ (here interpreting the simplex $\tilde{\alpha}$ as an element of a free group, or perhaps free groupoid). We note that it is obvious from this definition that the first diagram in the theorem commutes, as desired.\footnote{It may seem odd that we have managed to avoid the work done by tonks to define $b$, but in fact we have used it: $b$ is key to define $\text{St}_1$ in the first place. However, it is true that we will skip the proof that $ba = \id$ by working in our enriched category framework, though that proof is the easiest part of Tonks' work.}
	For objects $p,q$ of $\freeatom{\C}$, the hom-object $\C[p,q]$ has generated in degree $n$ by $k$-fold tensors $x_1 \otimes \cdots \otimes x_k$ of total degree $n$, with each $x_i$ an atom, and the list ``composable'' in that there is a list of objects $p = p_0,...,p_k = q$ such that $x_i \in [p_{i-1}, p_i]$. The composition map is defined on these generators by 
	$$(x_1 \otimes \cdots \otimes x_k) \circ (y_1 \otimes \cdots y_{k'}) = x_1 \otimes \cdots \otimes x_k \otimes y_1 \otimes \cdots y_{k'}$$
\begin{proposition}
	The functor $B$ takes a tensor $x_1 \otimes \cdots \otimes x_k$ of atoms to the product
	$$\prod_{\sigma \in S_{p_1,...,p_k}} (s_{\sigma_1}x_1 ,..., s_{\sigma_k} x_k)^{\text{sgn}(\sigma)}$$
\end{proposition}
\begin{proof}
	By definition, the tensor $x_1 \otimes \cdots \otimes x_k$ is the composition of $x_1$ through $x_k$. Since $B$ is a morphism in $\tensorcrossedcat$, it preserves compositions. Appealing to the definition of composition in the codomain, it must be sent by $B$ to
$$
b(b(\cdots b(x_1 \otimes x_2) \cdots \otimes x_{k-1}) \otimes  x_k)
$$
Expanding the definition of $b$ iteratively then give the result, along with the following construction: given $\sigma = (\sigma_1,\sigma_2) \in S_{p_1,p_2}$ and $\sigma' = (\sigma_{12},\sigma_4) \in S_{p_1 + p_2,p_3}$, we can produce an element $(\sigma_{12}\circ\sigma_1, \sigma_{12} \circ \sigma_2, \sigma_4) \in S_{p_1,p_2,p_3}$. Furthermore, this gives a bijection $S_{p_1,p_2} \times S_{p_1 + p_2, p_3} \to S_{p_1,p_2,p_3}$, and a similar construction gives a bijection $S_{p_1,p_2} \times S_{p_1 + p_2,p_3,...,p_k} \to S_{p_1,...,p_k}$, and these maps are multiplicative on sign.
\end{proof}
\subsection{The functor $A$}  
	The functor $A: \text{St}_1 \C$ will be slightly more complicated to define. Recall that the simplices in the hom objects of $\C$ are given as compositions 
	$$ (f_1 \cdot \alpha_1)\circ \cdots \circ (f_k \cdot \alpha_k )$$
	where each $\alpha_i$ is atomic and each $f_i$ is a degeneracy operator (and this decomposition is unique). 
	From here on we shall write an arbitrary simplex of $\C[p,q]$ as $(x_1,...,x_k)$, and mean by this that each $x_i$ is a degeneracy of an atomic simplex, and these $x_i$ are a composable chain and we are referring to their composition. Since the simplices generate the groups $\text{St}_1\C[p,q]_n$, we need to define the value of $A$ on each of these simplices. To extend the definition of $a$, we must in particular have that an $n$ simplex $(n \ge 3)$ given as a composition of atoms $\alpha \circ \beta$ should be sent to
	$$\prod_{i=0}^n \left( \alpha_{0\cdots i} \otimes \beta_{i \cdots  n}\right)^{x_0 \otimes y_{0i}}$$
More generally, but somewhat informally, for a general simplex $(x_1,...,x_k)$ we would like to define $A(x_1,...,x_k)$ as $a(a(\cdots a(x_1,x_2) \cdots, x_{k-1} ), x_k)$. This does not precisely typecheck since, despite writing things as tensors, we do not have any actual tensor products, but the formula obtained by expanding this expression out gives us our formula for $A$:
$$
A(x_1,...,x_k) = 
\begin{cases}
x_1 \otimes \cdots \otimes x_k & \text{if } (x_1,...,x_k) \in \C[p,q]_0 \\
\prod_{i = k}^1 (x_1)_0 \otimes \cdots \otimes x_i \otimes \cdots \otimes (x_k)_1 & \text{if } (x_1,\cdots,x_k) \in \C[p,q]_1 \\
\TODO{last case} & \text{ if} (x_1,...,x_k) \in \C[p,q]_2 \\
\sum_{D \in [n]_k} (x_1)_{D_1} \otimes \cdots \otimes (x_k)_{D_k} & \text{if } (x_1,...,x_k) \in \C[p,q]_n \text{ for } n \ge 3
\end{cases}
$$
\begin{proposition} 
	The map $A$ defined as above on hom-objects and as the identity on objects defines a morphism in $\tensorcrossedcat$.
\end{proposition}
\begin{proof}
	We must prove that each component is a map of crossed complexes, and that it commutes with composition. For the first point, the main thing to be done is to prove that the map commutes with boundary operators. This follows because Tonks already did it for things generated by two things and for the rest, it follows because our map is the same as apply tonks map k times, at least symbolically \TODO{try to write this better}. For the second point, \TODO{when you write this out diagramatically it becomes fairly nice and you see that while there appear to be a bunch more terms in one, they all have some degenerate thing involved. I should do that.}
\end{proof}
\subsection{The contracting homotopy}
	It is evident that $AB = \id$, as it sends each generator to itself. In this section we will provide, for every pair of objects $p$ and $q$ in $\C$, a contracting homotopy from the identity to $BA$ restricted to $\text{St}_1 \C[p,q]$. Again, we will essentially copy Tonks. We apologize in advance for the sheer number of indices that appear here. 
	
	A \emph{simplicial operator} of arity $k$ and degree $(m,n)$ is a finite formal sum
	$$G = \sum_{\alpha} \varepsilon_\alpha( \lambda^\alpha_1, \cdots, \lambda^\alpha_k )$$
	where each $\lambda^\alpha_i$ is a map $[n] \to [m]$ in $\Delta$. 
	We call $G$ \emph{degenerate} if for each $\alpha$, there is some $j^\alpha$ such that all the $\lambda^\alpha_i$ factor through the codegeneracy $s^{j^{\alpha}}: [n] \to [n-1]$. The \emph{derived operator} of $G$ is given by 
	$$G' = \sum_{\alpha} \varepsilon_\alpha( {\lambda^\alpha_1}', \cdots, {\lambda^\alpha_k}')$$
	Where for a map $\lambda: [n] \to [m]$, the map $\lambda': [n+1] \to [m+1]$ is the unique map such that $\lambda(0) = 0$ and $\lambda' \circ d^0 = d^0 \lambda$.
	Sums of operators are formed formally in the free abelian group, \textbf{or, if $n = 2$ they are formed in the free nonabelian group} and the products of operators are given as \TODO{write this.} A simplicial operator \textit{respects degeneracies} if $s^kG$ is degenerate for any $k$.
	\\\\
	A \emph{total simplicial operator of degree $(m,n)$} is a collection $\mathbb{G} = \{\mathbb{G}_k\}_{k \ge 1}$ of simplicial operators of degree $(m,n)$; one of each arity for $k\ge 2$. We say that $\mathbb{G}$ respects degeneracies if each $\mathbb{G}_k$ respects degeneracies. 

\begin{theorem}
	Let $\C$ be a simplicial computad, and $\mathbb{G}$ a total simplicial operator of degree $(m,n)$ which respects degeneracies. Suppose $m \ge 2$ and $n \ge 3$. Then for any objects $p,q$ of $\C$, there is a well defined morphism $\text{St}_1\C[p,q]_m \to \text{St}_1\C[p,q]_n$ of groups over the groupoid $\text{St}_1\C[p,q]_1$ , which is given on generators $(x_1, \cdots , x_k)$ via
$$\widetilde{\mathbb{G}}(x_1, \cdots , x_k) = \sum_\alpha \varepsilon_\alpha({\lambda^\alpha_1}^*x_1, \cdots , {\lambda^\alpha_k}^*x_k)^{\gamma_\alpha}$$
Where $\gamma_\alpha = ((x_1)_{0\lambda^\alpha_1(0)},...,(x_k)_{0\lambda^\alpha_k(0)})$ is there to make sure this product makes sense. 
\end{theorem}
\begin{proof}
	Since we are defining maps out of a free groups, it is relatively obvious that this is well-defined as a map of skeletal groupoids. So the main thing to be done is to show that it commutes with the action. But this too is immediate, since the action is also free.
\TODO{think about that more and explain it better}
\end{proof}
Having set ourselves up, we are ready to define the operators we are interested in: first, a standard one we shall use is the operator of arity $k$ and degree $(n,n-1)$
$$\mathds{D}^k_n = \sum_{i = 0}^n \left( (-1)^i d^i,...,(-1)^i d^i \right) = -(\mathds{D}^k_{n-1})' + (d^0, \cdots, d^0)$$
modelling the standard differential.
\begin{definition}
	We define simplicial operators $F^k_n$ of arity $k$ and degree $(n, n)$ as follows:
	$$F_0^k = (\id,\cdots, \id)$$
	$$F_1^k = (\id,s(0)d(1), \cdots, s(0)d(1)) + (s(0)d(0), \id, s(0) d(1),\cdots, s(0)d(1)) + \cdots + (s(0)d(0),\cdots,\id) $$
	\TODO{the correct ordering above depends on the correct ordering for $A$.}
	For $n \ge 2$, we define
	$$F_n^k = \prod_{p_1 + \cdots + p_k = n} \prod_{S_{p_1,\cdots, p_k}} \left( d[0\to p_1] s(\sigma_1), d[p_1 \to p_1 + p_2] s(\sigma_2), \cdots , d[p_1 + \cdots + p_{k-1} \to p_1 + \cdots + p_k] s(\sigma_k)  \right)$$
\end{definition}
Note that for varying $k$, these assemble into total simplicial operators of degree $(n,n)$. 
\begin{proposition}
	Let $\mathbb{F}_n = (F_n^k)_{k \ge 2}$, with $F_n^k$ as defined above. Then for $n \ge 3$, $\mathbb{F}_n$ defines maps $\widetilde{\mathbb{F}}_n$ which coincide with the composition $BA$ for any simplicial computad and any hom-object of that computad. Furthermore, we have $\mathds{D}^k_n \mathbb{F}^k_n = \mathbb{F}^l_{n+1}D^k_n $, and $\mathbb{F}^k_n$ respects degeneracies.
\end{proposition} 
\begin{proof}
	The fact that these model our composition $BA$ is immediate from the formulas given for $B$ and $A$.  \TODO{consider the $n=2$ term.}
	Let us prove that these operators respect degeneracies: let $s^k : [n] \to [n+1]$ be a codegeneracy. Consider the term
	$$ \left( d[0\to p_1] s(\sigma_1), d[p_1 \to p_1 + p_2] s(\sigma_2), \cdots , d[p_1 + \cdots + p_{k-1} \to p_1 + \cdots + p_k] s(\sigma_k)  \right)$$
	This term can be represented in the language of \TODO{add in the grid description earlier} as an $k\times n$ matrix 
	$$
		\begin{pmatrix}
			0 & \cdots & p_1 \\
			p_1 & \cdots & p_1 + p_2 \\
			\vdots & \ddots & \vdots \\
			p_1 + \cdots + p_{k-1} & \cdots & p_1 + \cdots + p_k
		\end{pmatrix}
	$$
	Where each row is a weakly increasing sequence and in every column exactly one row increases. The effect of postcomposing with the codegeneracy $s^j$ is that every term strictly greater than $j$ is decreased by $1$. It follows that in the resulting grid representation of 
	$$ s^j \left( d[0\to p_1] s(\sigma_1), d[p_1 \to p_1 + p_2] s(\sigma_2), \cdots , d[p_1 + \cdots + p_{k-1} \to p_1 + \cdots + p_k] s(\sigma_k)  \right)$$ 
	there will be a column in which every row is constant (the column in which previously, some row increased from $j$ to $j+1$), and therefore the $c$ of this column gives a degeneracy $s^c$ which each row factors through, proving that the element is degenerate, as desired.  
	\\
	\indent It remains to show the relationship with $\mathds{D}^k_n$. Morally, this follows from this modelling our map $BA$, but for completeness we will show it explicitly here. First, let us examine the effect of precomposing one of the terms in $\mathbb{F}^k_{n+1}$ with $d^j:[n] \to [n+1]$: it takes the $k \times (n+1)$ grid
	$$
		\begin{pmatrix}
			0 & \cdots & p_1 \\
			p_1 & \cdots & p_1 + p_2 \\
			\vdots & \ddots & \vdots \\
			p_1 + \cdots + p_{k-1} & \cdots & p_1 + \cdots + p_k
		\end{pmatrix}
	$$
	to the $k \times n$ grid obtained by omitting the $j$th column. In the other direction, postcomposing a $k \times n$ grid
	$$
		\begin{pmatrix}
			0 & \cdots & p_1 \\
			p_1 & \cdots & p_1 + p_2 \\
			\vdots & \ddots & \vdots \\
			p_1 + \cdots + p_{k-1} & \cdots & p_1 + \cdots + p_k
		\end{pmatrix}
	$$
	with $d^j: [n] \to [n+1]$ has the effect of adding $1$ to every entry $\ge j$. To match these terms up, we note that the terms of the form $d^j \mathbb{G}$ can also be written in the form $\mathbb{G}' d^\ell$, where $\ell$ is the column number in which the index $j$ first appears, and $\mathbb{G}'$ is $\mathbb{G}$ but with a column inserted at position $\ell$, which increases only in the row where $j$ first appears in $\mathbb{G}$. 
	However, there are additional terms of the form $\mathbb{G} d^\ell$: the terms in which no row of $\mathbb{G}$ increases twice in a row at index $\ell$ cannot be written as $d^j \mathbb{G}'$. 
	However, we note that these terms all cancel out with each other; for any $\mathbb{G} d^\ell$ where no row increases twice in a row at index $\ell$, we have that the grid corresponding to $\mathbb{G} d^\ell$ must have some column where two rows $r_1$ and $r_2$ both increase. 
	Then in $\mathbb{G}$ the rows $r_1$ and $r_2$ increase at columns $\ell$ and $\ell + 1$, and if we let $\tilde{\mathbb{G}}$ be the same as $\mathbb{G}$ except that $r_2$ increases at $\ell$ and $r_1$ increases at $\ell + 1$, then we have that $\tilde{\mathbb{G}}d^\ell = \mathbb{G} d^\ell$, but they have opposite signs and so the terms of this form cancel out. With this observation, careful bookkeeping of the signs results in the sums being the same.
\end{proof}
We are now ready to define the simplicial operator that will give us the homotopy we want:
\begin{definition}
	Define simplicial operators $\Phi^k_n$ of arity $k$ and degree $(n,n+1)$ as follows:
	$$\Phi_0 ^k= (s(0),...,s(0))$$
	$$\Phi_n^k = -\Phi_{n-1}' + s(0)(F_n^k)'$$
\end{definition}
	This defines our derivation $\phi$ for degrees $n \ge 2$. For degree $1$ we have essentially the same formula, but must be careful about the order: to match with the order of $BA$, we define 
	\begin{align*}
		\phi(x_1,...,x_k) =  &(s_0(x_1),s_1(x_2),...,s_1(x_k)) \cdot \\
							 &(s_0s_0d_0(x_1),s_1(x_2),s_2(x_3),...,s_2(x_k)) \cdot \\
							 &\cdots \\
							 &(s_0s_0d_0 (x_1), \cdots ,s_0s_0d_0(x_{k-2}), s_0(x_{k-1}), s_1(x_k))
	\end{align*}
\section{Appendix: Explicit description of Left Adjoint $\tensorcrossedcat \to \cartcrossedcat$}
	\subsection{The more difficult functor}
		We now define the adjunction that will by the equivalence with $(\omega,1)-\text{Cat}$.be the most involved to define, as well as the most difficult to analyze: the adjunction 
		$$\tensorcrossedcat \rightleftarrows \cartcrossedcat$$
		One direction of this is much easier than the other: 
		\begin{definition}
			The functor $\forgetcartesian : \cartcrossedcat \to \tensorcrossedcat$ is induced by the identity functor on $\stinfty$, which is naturally a lax monoidal functor (from the cartesian monoidal structure to the gray monoidal structure). 
		\end{definition}
		Spelled out more explicitly, this means that for $\C \in \cartcrossedcat$, $\forgetcartesian(\C)$ has objects and hom-objects the same as those of $\C$. The composition operation in $\forgetcartesian(\C)$ is the composition
		$$\C[a,b] \otimes \C[b,c] \to \C[a,b] \times \C[b,c] \to \C[a,c]$$
		Where the first map is induced by the fact that $\otimes$ gives a semicartesian monoidal structure, and the second map is the composition in $\C$. On morphisms, $\forgetcartesian$ leaves functors unchanged.
		\\
		\indent We now move to describing the adjoint $\forcecartesian$. 
		Before embarking on the general, we note a few particular instances: 
		Let $G,H \in \stinfty$, let $\C_G \in \graycatzero$ have $\ob(C_G) = {x,y}$ and 
		$$\C_G[x,y] = G \quad \C_G[x,x] = * \quad \C_G[y,y] = * \quad \C_G[y,x] = \emptyset$$
		and let $\C_H \in \cartesiancatzero$ have $\ob(C_H) = {y,z}$ and 
		$$\C_H[y,z] = H \quad \C_H[y,y] = * \quad \C_H[z,z] = * \quad \C_H[z,y] = \emptyset$$
		With compositions being identities in every case.
		Now, we note that for $\forcecartesian$ to be the left adjoint we are looking for, it is necessary that
		$$\Hom_{\cartesiancatzero}(\forcecartesian(\C_G), \C_H) = \Hom_{\graycatzero}(\C_G, \forgetcartesian(\C_H)) $$
		Which suggests that $\forcecartesian(C_G)$ should simply be $C_G$ (or rather, should have the same objects and hom-objects, with compositions the identity in every case). Combining this with the fact that left adjoints must preserve pushouts, we get an example that will guide our definition: Let $\C_{GH} \in \graycatzero$ be defined as $\ob(\C) = {x,y,z}$, where
		$$\C[x,y] = G \quad \C[y,z] = H \quad \C[x,z] = G \otimes H \C[x,x] = * \quad \C[y,y] = * \quad \C[z,z] = *$$
		And the other three hom-objects empty (that is, they are the empty $\omega$-groupoid). The compositions we can define to be identity maps in every case. Then $\C_{GH}$ is the pushout of $\C_G$ and $\C_H$ along the inclusion of the object $y$ into each, and hence $\forcecartesian(\C_{GH})$ must also be the pushout of $\C_G$ and $\C_H$: this leaves everything unchanged except that $\forcecartesian(\C_{GH})[x,z] = G \times H$. 
		So the affect of $\forcecartesian$ is to quotient out hom-spaces in the same way that $G \otimes H \to G \times H$ is quotiented. 
		\\\\
		Now let us make this all a bit more precise:
		\begin{definition}
			We define $\forcecartesian$ as follows: for $\C \in \graycatzero,$ $\ob(\forcecartesian(\C)) = \ob(\C)$. On hom-objects, (see next section).
		\end{definition}
\subsection{Products of Crossed Complexes}
		We will analyze this map using crossed complexes, as they provide an easier generators-and-relations type object to work with. For $C,D \in \crcom$, their categorical product $C \times D$ is given by $(C \times D)_n = C_n \times D_n$, with all structure maps defined componentwise. For $(k,\ell) \in C_1 \times D_1)$ and $(c,d) \in C_n \times D_n$, the action is given by the actions on the individual components, that is
	$$(c,d)^{(k,\ell)} = (c^k,d^\ell)$$
I suppose there isn't much more to say here...
\subsection{tensor products of crossed complexes}		
	We will analyze this map using crossed complexes, as they provide an easier generators-and-relations type object to work with. 
For $C,D \in \crcom$, their tensor product is presented as follows:\footnote{see the tonks eilenberg zilber paper} 
it has generators given by $c_m \otimes d_n \in (C \otimes D))_{m+n}$ for $c_m \in C_m$	and $d_n \in D_n$. The source and target maps are given componentwise. 
For now, let us ignore the actions because those are annoying. 
The relations to generate the group structure are as follows:
$$
(c_mc'_m \otimes d_n) = 
\begin{cases}
	c_m \otimes d_n \cdot c'_m\otimes d_n & \text{if }m = 0 \text{ or } n \ge 2 \\
	c'_m \otimes d_n \cdot (c_m \otimes d_n)^{c'_m \otimes sd_n} & \text{if } n = 1 \text{ and } m \ge 1
\end{cases}
$$
Symmetrically, we have
$$
(c_m \otimes d_nd'_n) = 
\begin{cases}
	c_m \otimes d_n \cdot c_m\otimes d'_n & \text{if } n = 0 \text{ or } m \ge 2 \\
	(c_m \otimes d_n)^{sc_m \otimes d_n'} \cdot (c_m \otimes d'_n) & \text{if } n = 1 \text{ and } m \ge 1
\end{cases}
$$
$$\del (c_m \otimes d_n) = \del(c_m) \otimes d_n \cdot c_m \otimes \del(d_n)$$
Except in the cases where the degree(s) are $1$. If $m = 1$, replace the first term with
$$(sc_1 \otimes d_n)^{-1} \otimes (tc_1 \otimes d_n)^{c_1 \otimes sd_n}$$
and similarly for the case $n = 1$.
If $m = n = 1$, then in particular 
$$\del(c_1 \otimes d_1) = (sc_1 \otimes d_1)^{-1} \cdot (c_1 \otimes td_1)^{-1} \cdot (tc_1 \otimes d_1) \cdot (c_1 \otimes sd_1) $$
There is a natural map from $C \otimes D \to C \times D$ which is described on generators as being nonzero on exactly those generators of the form $c_0 \otimes d_n$ or $c_m \otimes d_0$, which it sends to $(0, d_n)$ or $(c_m, 0)$, respectively. We shall refer to this map as $\pi$.
\begin{lemma}
 	The natural map $\pi: C \otimes D \to C \times D$ has kernel consisting generated by those elements which vanish for degree reasons, as well as those elements of the form $c_1 d_1 c_1^{-1}d_1^{-1}$ and $c_2d_2c_2^{-1}d_2^{-1}$.
\end{lemma}
\begin{proof}
	Firstly, it is clear that the map $\pi$ is an isomorphism on objects. In degree $1$, we have that 
	$$\pi(c_1d_1 \cdots c_nd_n) = \pi(c_1'd_1' \cdots c_k'd_k') \iff c_1\cdots c_n = c_1' \cdots c_k \; \text{and}\; d_1 \cdots d_n = d_1' \cdots d_k'$$
	In this case, we have that by multiplying by commutators of appropriate elements, we get these elements equal, as desired.
	\\\\
	In degree 2, the presentation given by another thing gives a similar computation.
	In degree 3 and above, this becomes trivial.
\end{proof}
Our goal is to define a functor $\tensorcrossedcat \to \cartcrossedcat$ by quotienting out by various relations imposed by the the composition operation and $\pi$, as described in the previous sections with $\omega$-groupoids. So, let $\C \in \tensorcrossedcat$. 
We define $\tencart{\C}$ to have the same objects in $\C$, and have hom sets as follows: for $a,c \in \ob(\C)$, we let 
$$
\tencart{\C}[a,c]_0 = \C[a,b]_0
$$
$$\tencart{\C}[a,b]_n = \C[a,b]_n/K$$
Where $K$ is the subgroup of $\C_n$ generated by the union over all tuples $r_1,...,r_n \in \ob(C)$ of the image under composition of the kernel of the maps $\C[a,r_1] \otimes \cdots \otimes \C[r_n,b] \to \C[a,r_1] \times \cdots \times \C[r_n,b]$. We must prove that this defines a functor and that this functor gives an adjoint to the inclusion. 
So first, let us prove that this actually defines a functor. 
We have not yet finished defining this functor's action on objects: we have described the objects and morphisms of $\tencart{\C}$, but not the composition operation. To define the composition operation, we must define maps 
$$g: \tencart{\C}[a,b] \times \tencart{\C}[b,c] \to \tencart{\C}[a,c]$$
On objects, we can let this be the same as $\circ$. On higher cells, we define $g([a],[b]) = [\circ(a\otimes s(b) \cdot s(a) \otimes b)]$. 
We must prove that this is independent of the choice of representatives $a$ and $b$. Let us suppose that $[a] = [a']$. 
Then $a$ and $a'$ differ by some combination of elements in the images of kernels of maps $\C[a,r_1] \otimes \cdots \otimes \C[r_n,b] \to \C[a,r_1] \times \cdots \times \C[r_n,b]$. We write
$$a^{-1}a' = k_1 \cdots k_m$$
and then note that this gives us that 
\begin{align*}
	\circ(a \otimes s(b))^{-1} \circ(a' \otimes s(b)) &= \circ(a^{-1} a' \otimes s(b)) \\
	&= \circ(k_1 \cdots k_m \otimes s(b)) \\
	&= \circ(k_1 \otimes s(b)) \cdots \circ (k_m \otimes s(b)) \\
\end{align*}
By lemma \TODO{write this lemma} we have that $\pi(k_1 \otimes s(b)) = 0$, and hence by definition $\circ(k_1 \otimes s(b))$ is $0$ in $\tencart{\C}[a,c]$, as desired. 
%
% Let us suppose that $a$ and $a'$ differ by just one such element. That is, we are supposing that th
%ere is an $r \in \ob(\C)$ and a $k \in \C[a,r]\otimes \C[r,b]$ such that $\pi(k) = 0$ and $\circ(k)=a^{-1}a$. We wish to prove from this that 
%$$[\circ(a\otimes s(b) \cdot s(a) \otimes b)] = [\circ(a'\otimes s(b) \cdot s(a') \otimes b)]$$.
%Since $s(a) = s(a')$, this comes down to proving that 
%$$[\circ(a \otimes s(b))] = [\circ(a' \otimes s(b))]$$
%Equivalently, 
%$$[\circ(a^{-1}a' \otimes s(b))] = 0$$
%And this follows, as 
%$$[\circ(a^{-1}a' \otimes s(b))] = []$$
%\TODO{finish this bit of the argument - note that even this is only for the part where they differ by a single thing. it should follow by the description of the kernel we gave - since $r$ goes to zero, it is either commutators or bad dimension, both of which are preserved by tensoring with a point.}
%A symmetric argument works for the right hand side of the tensor product.
%Having defined the composition operation, associativity and identity follow readily from associativity and identity of the composition operation in $\C$.
\\\\
It remains to show that this defines an adjoint.
So, we wish to show that for $\C \in \tensorcrossedcat$ and $\D \in \cartcrossedcat$, we have a natural isomorphism of hom-sets
$$
\cartcrossedcat[\tencart{\C}, \D] \cong \tensorcrossedcat[\C,U(\D)]
$$ 
Let $F: \C \to U(\D)$ be a functor. We shall define $F^\flat : \tencart{\C} \to \D $ as follows: on objects, it is the same as $F$. To define this on hom objects, we must show that for $a,b \in \ob(\tencart{C})$ we can fill in the dashed arrow in the diagram
\begin{center}
\begin{tikzcd}[sep = large]
	\C[a,b] \ar[d]\ar[r, "F"] & \D[Fa, Fb] \\
	\tencart{\C}[a,b] \ar[ru, dashed]
\end{tikzcd}
\end{center}
As $\C$ is levelwise a quotient, to do this it suffices to show that the elements we quotient out by are necessarily sent to identities by $F$. Let $a,b,c \in \ob\C$ and $k \in \C[a,b] \otimes \C[b,c]$ with $\pi(k) = 0$. 
By functoriality of $F$, the diagram
\begin{center}
\begin{tikzcd}
	\C[a,b] \otimes \C[b,c] \ar[rr] \ar[dd] \ar[rd, "\pi"] & & \D[Fa,Fb] \otimes \D[Fb,Fc] \ar[d] \\
	& \C[a,b] \times \C[b,c] \ar[r] & \D[Fa,Fb] \times \D[Fb, Fc] \ar[d] \\
	\C[a,c] \ar[rr] & & \D[Fa, Fc] 
\end{tikzcd}
\end{center}
commutes. Hence if $\pi(k) = 0$, then $F(\circ(k)) = 0$, as desired. \TODO{why does the horizontal map in the middle exist? can we define it as $(a,b) \mapsto \pi(F(a \otimes s(b) + s(b) \otimes a))$?}
\\\\
For the other direction: Given $G: \tencart{\C} \to \D$, we must define $G^\sharp: \C \to U(\D)$. Again, we let $G^\sharp$ on objects be the same as on $G$. For the morphisms, we simply let $G^\sharp: \C[a,b] \to U(\D)[Ga,Gb]$ be the composition
\begin{center}
\begin{tikzcd}
	\C[a,b] \ar[r] & \tencart{\C}[a,b] \ar[r, "G"] & \D[Ga,Gb]
\end{tikzcd}
\end{center}
And we are done. \TODO{prove these are mutually inverse?}.
\\\\
Having given the adjunction, we would like to investigate its homotopical properties; ideally proving it is a Quillen adjunction. Unfortunately, we have no Quillen model structure on $\tensorcrossedcat$. Fortunately, we are not interested in this adjunction so much as we are interested in the composite adjunction
$$
	\ssetcat \leftrightarrows \cartcrossedcat
$$ 
And we can show that this is a Quillen adjunction rather readily:
\begin{theorem}
	The adjunction $$ \ssetcat \leftrightarrows \cartcrossedcat$$ given by composing BLAH and BLAH is a quillen adjunction.
\end{theorem}
\begin{proof}
	We will show that the right adjoint preserves weak equivalences and fibrations. For weak equivalences: let $F: \C \to \D$ be a weak equivalence in $\cartcrossedcat$. Then by BLAH it induces an equivalence of homotopy categories and equivalences of hom-objects. By BLEH $U(F)$ induces an equivalence of homotopy categories, and by [the previous stuff paper stuff] $U(F)$ induces weak equivalences of hom-objects.
	\\\\
	Now, for the fibrations: let $F \C \to \D$ be a fibration in $\cartcrossedcat$. By [BLAH] it induces an isofibration on homotopy categories, and is a fibration on each of the hom-objects. By [the fact that the right adjoint is hom-object-wise realization] this gives the thingy we want.
\end{proof}

\newpage

\end{document}
